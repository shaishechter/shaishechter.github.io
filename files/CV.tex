\documentclass[leqno,11pt,a4paper]{amsart}
%\input{general}

\usepackage{array,float,hyperref,yhmath, tikz-cd}
\usepackage{amssymb, amsmath, amsthm, amsbsy, amscd, mathrsfs,stmaryrd}
\usepackage{wrapfig}
\usepackage{fullpage}
\usepackage{graphicx}
\usepackage{hyperref}
\usepackage{charter}
\usepackage{tikz}
\usetikzlibrary{arrows}
\urlstyle{rm}
\usepackage{longtable}


\usepackage[
	maxnames=10,
	firstinits=true,
	backend=biber, 
	url=false, doi=false, isbn=false,
	style=authortitle, citestyle=verbose-trad1]{biblatex}
 	 \AtBeginBibliography{\scriptsize}
 
\addbibresource{MarieCurie.bib}
%
%\usepackage{apacite}

\usepackage[left=1.5cm, right=1.5cm, top=1.6cm, bottom=1.7cm]{geometry}
\usepackage{times}
\usepackage{lastpage}

\usepackage{stmaryrd}
\usepackage{fancyhdr}


\hypersetup{
  colorlinks=true,
  citecolor=black,
  linkcolor=black,
  urlcolor=black}
 
\usepackage{setspace}
\onehalfspacing


\numberwithin{equation}{section}
\numberwithin{figure}{section}

\newcommand{\rev}[1]{{\color{black}{#1}}}

% adjust row height in tables 
% 
\setlength{\extrarowheight}{0.05cm} 

% theorems etc

\theoremstyle{plain}
 \newtheorem{theorem}{Theorem}[section]
 \newtheorem{proposition}[theorem]{Proposition}
 \newtheorem{lemma}[theorem]{Lemma}
 \newtheorem{corollary}[theorem]{Corollary}
 \newtheorem{condition}{Condition}[section]
 \newtheorem{goal}{Goal}
 \newtheorem{conjecture}[theorem]{Conjecture}
 \newtheorem{thmABC}{Theorem}
 \renewcommand{\thethmABC}{\Alph{thmABC}}

\theoremstyle{definition}
 \newtheorem{definition}[theorem]{Definition}
 \newtheorem*{notation}{Notation}

\theoremstyle{remark}
 \newtheorem{assumption}{Assumption}[section]
 \newtheorem*{claim}{Claim}
% \newtheorem{condition}{Condition}[section]
 \newtheorem*{observation}{Observation}
 \newtheorem{remark}[theorem]{Remark}
 \newtheorem*{acknowledgements}{Acknowledgements}
 \newtheorem{question}[theorem]{Question}
 \newtheorem{example}[theorem]{Example}


% various newcommands, abreviations, etc
%

\newcommand{\car}{\curvearrowright}
\newcommand{\nl}{\triangleleft}
\newcommand{\nlgr}{\triangleleft_{\textup{gr}}}

\newcommand{\be}{\begin{enumerate}}
\newcommand{\ee}{\end{enumerate}}

\newcommand{\brem}{\begin{remark}}
\newcommand{\erem}{\end{remark}}

\newcommand{\bp}{\begin{proof}}
\newcommand{\ep}{\end{proof}}

\newcommand{\F}{\ensuremath{\mathbb{F}}}
\newcommand{\Fp}{\ensuremath{\mathbb{F}_p}}
\newcommand{\Fq}{\ensuremath{\mathbb{F}_q}}
%\newcommand{\Fqf}{\ensuremath{\mathbb{F}_{q^f}}}
\newcommand{\Fpf}{\ensuremath{\mathbb{F}_{p^f}}}
\newcommand{\N}{\ensuremath{\mathbb{N}}}
\newcommand{\Zp}{\ensuremath{\mathbb{Z}_p}}
\newcommand{\Qp}{\ensuremath{\mathbb{Q}_p}}
\newcommand{\pP}{\ensuremath{\mathbb{P}}}
\newcommand{\C}{\ensuremath{\mathbb{C}}}
\newcommand{\R}{\ensuremath{\mathbb{R}}}
\newcommand{\Q}{\ensuremath{\mathbb{Q}}}
\newcommand{\Z}{\ensuremath{\mathbb{Z}}}


\newcommand{\bfo}{{\bf 1}}
\newcommand{\bfz}{{\bf 0}}

\newcommand{\bfbeta}{\boldsymbol{\beta}}
\newcommand{\bflambda}{\boldsymbol{\lambda}}
\newcommand{\bfphi}{\boldsymbol{\phi}}
\newcommand{\bfa}{\ensuremath{\mathbf{a}}}
\newcommand{\bfb}{\ensuremath{\mathbf{b}}}
\newcommand{\bfe}{\ensuremath{\mathbf{e}}}
\newcommand{\bfE}{\ensuremath{\mathbf{E}}}
\newcommand{\bff}{\ensuremath{\mathbf{f}}}
\newcommand{\bfF}{\ensuremath{\mathbf{F}}}
\newcommand{\bfg}{\ensuremath{\mathbf{g}}}
\newcommand{\bfl}{\ensuremath{\mathbf{l}}}
\newcommand{\bfm}{\ensuremath{\mathbf{m}}}
\newcommand{\bfn}{\ensuremath{\mathbf{n}}}
\newcommand{\bfr}{\ensuremath{\mathbf{r}}}
\newcommand{\bfs}{\ensuremath{\mathbf{s}}}
\newcommand{\bfS}{\ensuremath{\mathbf{S}}}
\newcommand{\bfw}{{\boldsymbol{w}}}
\newcommand{\Frcq}{F_{r,c}(\mathbb{F}_q)}
\newcommand{\frcq}{\mathfrak{f}_{r,c}(\mathbb{F}_q)}

% boldsymbol looks better than mathbf ... !?

\newcommand{\bfx}{{\boldsymbol{x}}}
\newcommand{\bfy}{\ensuremath{\mathbf{y}}}
%\newcommand{\bfz}{\ensuremath{\mathbf{z}}}

\newcommand{\bfB}{\ensuremath{\boldsymbol{B}}}
\newcommand{\bfG}{\ensuremath{\mathbf{G}}}
\newcommand{\bfH}{\ensuremath{\mathbf{H}}}
\newcommand{\bfR}{\ensuremath{\mathbf{R}}}
\newcommand{\bfU}{\ensuremath{\mathbf{U}}}
\newcommand{\bfX}{\ensuremath{\mathbf{X}}}
\newcommand{\bfY}{\ensuremath{\mathbf{Y}}}

\newcommand{\mcB}{\mathcal{B}}
\newcommand{\mcD}{\mathcal{D}}
\newcommand{\mcE}{\mathcal{E}}
\newcommand{\mcH}{\mathcal{H}}
\newcommand{\mcK}{\mathcal{K}}
\newcommand{\mcO}{\mathcal{O}}
\newcommand{\mcP}{\mathcal{P}}
\newcommand{\mcR}{\mathcal{R}}
\newcommand{\mcRone}{\ensuremath{\mathcal{R}}^{(1)}}
\newcommand{\mcM}{\mathcal{M}}
\newcommand{\mcN}{\ensuremath{\mathcal{N}}}
\newcommand{\mcV}{\ensuremath{\mathcal{V}}}

\newcommand{\calA}{\ensuremath{\mathcal{A}}}
\newcommand{\calH}{\ensuremath{\mathcal{H}}}
\newcommand{\mcL}{\ensuremath{\mathcal{L}}}
\newcommand{\calV}{\ensuremath{\mathcal{V}}}
\newcommand{\calW}{\ensuremath{\mathcal{W}}}

\newcommand{\mfa}{\ensuremath{\mathfrak{a}}}
\newcommand{\mff}{\ensuremath{\mathfrak{f}}}
\newcommand{\mfg}{\ensuremath{\mathfrak{g}}}
\newcommand{\mfh}{\ensuremath{\mathfrak{h}}}
\newcommand{\mfk}{\ensuremath{\mathfrak{k}}}
\newcommand{\mfgp}{\ensuremath{\mathfrak{g}'}}
\newcommand{\mfu}{\ensuremath{\mathfrak{u}}}
\newcommand{\mfv}{\ensuremath{\mathfrak{v}}}
\newcommand{\mfz}{\ensuremath{\mathfrak{z}}}

\newcommand{\SA}{\ensuremath{{S}}}
\newcommand{\SB}{\ensuremath{{T}}}

\newcommand{\rf}{\ensuremath{{\bf k}}}
\newcommand{\Rf}{\ensuremath{{\bf K}}}

\newcommand{\fieldK}{\ensuremath{{\mathbb K}}}

\newcommand{\set}[1]{\lbrace#1\rbrace}
\newcommand{\abs}[1]{\left|#1\right|}
\newcommand{\norm}[1]{\left\|#1\right\|}
\newcommand{\pmat}[1]{\begin{pmatrix}#1\end{pmatrix}}
\newcommand{\rmat}[1]{\left[\begin{matrix}#1\end{matrix}\right]}


\newcommand{\rmN}{\textup{N}}
\newcommand{\tud}{\textup{d}}
\newcommand{\upL}{\ensuremath{\textup{L}}}
\newcommand{\upR}{\ensuremath{\textup{R}}}
\newcommand{\rk}{\ensuremath{{\rm rk}}}
\newcommand{\nul}{\ensuremath{{\rm null}}}
\newcommand{\ol}{\ensuremath{\overline}}
\newcommand{\ul}{\ensuremath{\underline}}

\newcommand{\disjunion}{\ensuremath{\; \dot{\cup} \;}}

\newcommand{\Ct}{\mathbb{C}\llbracket T \rrbracket}
\newcommand{\e}{\ensuremath{\gamma}}
\newcommand{\sym}{\ensuremath{\mathcal{S}}}
\newcommand{\zirr}{\ensuremath{\zeta^{\rm irr}}}
\newcommand{\zcc}{\ensuremath{\zeta^{\rm cc}}}
\newcommand{\Irr}{\ensuremath{\text{Irr}}}
\newcommand{\wt}{\ensuremath{\widetilde}}
\newcommand{\wtm}{\ensuremath{\widetilde{m}}}
\newcommand{\wtB}{\ensuremath{\widetilde{B}}}
\newcommand{\mcC}{{\mathcal{C}}}
\newcommand{\tiIrr}{\widetilde{\Irr}}
\newcommand{\tir}{\tilde{r}}

\newcommand{\tr}{\textup{t}}
\newcommand{\lri}{\mathfrak o}
\newcommand{\Lri}{\mathfrak O}
\newcommand{\lfi}{\mathfrak k}
\newcommand{\la}{\langle}
\newcommand{\ra}{\rangle}

\newcommand{\smallgri}{\ensuremath{{\scriptscriptstyle \mathcal{O}}}}
\newcommand{\gri}{\ensuremath{{\scriptstyle \mathcal{O}}}}
\newcommand{\Gri}{\ensuremath{\mathcal{O}}}

\newcommand{\wtdelta}{\widetilde{\delta}}
\newcommand{\mfp}{\mathfrak{p}}
\newcommand{\mfP}{\mathfrak{P}}
\newcommand{\fS}{\mathcal{S}}
\newcommand{\hallgenerators}{\Delta}
\newcommand{\hallbasis}{\mathcal{H}}
\newcommand{\wh}{\widehat}
\newcommand{\rarr}{\rightarrow}
\newcommand{\kk}{\ensuremath{\mathbf{k}}}
\newcommand{\bj}{\overline{\jmath}}
\newcommand{\sll}{\mathfrak{sl}}
\newcommand{\nch}{\varepsilon_n}
\newcommand{\udots}{\mathinner{\mskip1mu\raise1pt\vbox{\kern7pt\hbox{.}}
\mskip2mu\raise4pt\hbox{.}\mskip2mu\raise7pt\hbox{.}\mskip1mu}}
\newcommand{\bigdotcup}{\ensuremath{\mathop{\dot{\bigcup}}}}

\newcommand{\level}{N}

\DeclareMathOperator{\n}{n}
\renewcommand{\r}{\tilde{r}}
\DeclareMathOperator{\nega}{neg}
\DeclareMathOperator{\nsp}{nsp}
\DeclareMathOperator{\INeg}{INeg}
\DeclareMathOperator{\Neg}{Neg}
%\DeclareMathOperator{\nch}{\varepsilon_n}
\DeclareMathOperator{\rmaj}{rmaj}
\DeclareMathOperator{\des}{des}
\DeclareMathOperator{\gp}{gp}
\DeclareMathOperator{\Des}{Des}
\DeclareMathOperator{\val}{val}
\DeclareMathOperator{\Spec}{Spec}
\DeclareMathOperator{\red}{red}
\DeclareMathOperator{\Fil}{Fil}
\DeclareMathOperator{\Ann}{Ann}
\DeclareMathOperator{\Cent}{Cent}
\DeclareMathOperator{\SubMod}{SubMod}
\DeclareMathOperator{\SL}{SL}
%\DeclareMathOperator{\sll}{sl}
\DeclareMathOperator{\gll}{gl}
\DeclareMathOperator{\End}{End}
\DeclareMathOperator{\inv}{inv}
\DeclareMathOperator{\invv}{inv}
\DeclareMathOperator{\diag}{diag}
\DeclareMathOperator{\Pf}{Pf}
\DeclareMathOperator{\Hom}{Hom}
\DeclareMathOperator{\cha}{char}
\DeclareMathOperator{\cut}{cut}
\DeclareMathOperator{\disc}{disc}
\DeclareMathOperator{\Ig}{Ig}
\DeclareMathOperator{\IG}{IG}
\DeclareMathOperator{\Id}{Id}
\DeclareMathOperator{\id}{id}
\DeclareMathOperator{\Rad}{Rad}
\DeclareMathOperator{\Stab}{Stab}
\DeclareMathOperator{\Sp}{Sp}
\DeclareMathOperator{\Ad}{Ad}
\DeclareMathOperator{\Ind}{Ind}
\DeclareMathOperator{\cc}{cc}
\DeclareMathOperator{\ad}{ad}
\DeclareMathOperator{\im}{im}
\DeclareMathOperator{\Mat}{Mat}
\DeclareMathOperator{\Alt}{Alt}
\DeclareMathOperator{\Sym}{Sym}
\DeclareMathOperator{\weight}{wt}
\DeclareMathOperator{\GL}{GL}
%\DeclareMathOperator{\Iden}{Id}
\DeclareMathOperator{\dl}{dl}
\DeclareMathOperator{\cd}{cd}
\DeclareMathOperator{\cs}{cs}
\DeclareMathOperator{\ch}{ch}
\DeclareMathOperator{\dep}{dep}
\DeclareMathOperator{\topo}{top}
\DeclareMathOperator{\trans}{tr}
\DeclareMathOperator{\Tr}{Tr}
% appearance of greek letters
%
\renewcommand{\epsilon}{\varepsilon}
\renewcommand{\phi}{\varphi}
%\renewcommand{\mid}{:}

%\pagenumbering{gobble}

\newcommand{\acronym}{RGNilPosChar}
\newcommand{\type}{MSCA-IF-EF-ST}




\begin{document}


%% Start page
\renewcommand{\thepart}{B-2}
~

%%Frontmatter
%\setcounter{page}{11}
\section*{Curriculum Vitae\\Shai Shechter}
%\textsf{Change all dates to (dd/mm/yyyy)}
\subsection*{\color{white}w }
\subsection*{\sc General Information}~
\begin{longtable}[H]{p{\textwidth} }
\textbf{Name}: Shai Shechter
\\
\textbf{Date and place of birth}: Haifa, Israel
\\
\textbf{Citizenship}: Israeli
\\
\textbf{Languages}: Hebrew (native), English (fluent), German (upper-intermediate)
\\
\textbf{Marital Status}: Married +1
\\
\textbf{ Corresponding Address}:  Faculty of Mathematics and Computer Science,  The Weizmann Institute of Science

 \hskip3.35cm~ 234 Herzl Street,  Rehovot, 7610001, Israel
%\item\textbf{Phone Number}: +972-507131339.
\\
\textbf{E-mail}: \url{shais1985@gmail.com} 
\\
\textbf{Website}: \url{https://shaishechter.github.io}
\end{longtable}
~\\
\subsection*{\sc Professional experience}~
\begin{longtable}{p{0.22\textwidth} p{0.2\textwidth} p{0.54\textwidth}}

 1/11/2019 -- Present day& Teaching assistant& \textbf{Faculty of Mathemtics and Computer Science, WIS,} Generalized functions,
\\
20/10/2018 -- Present day& Postdoctoral Fellow&\textbf{Faculty of Mathemtics and Computer Science, WIS}\\
1/2/2019 -- 30/6/2019& Teaching assistant&
\textbf{Gateway to academy preparatory school for applicants to engineering degrees from Bedouin communities in the Negev}, Real analysis I\\
1/10/2011 -- 30/2/2019 & Teaching assistant& \textbf{Department of Mathematics, BGU}, Real analysis I \& II, Algebra I, Discrete Mathematics
\\
1/7/2017 -- 30/7/2017& Undergraduate thesis reviewer&\textbf{Ministry of Education}\\
1/11/2016 -- 30/1/2017& Lecturer &\textbf{
Department of Education, BGU}, Pre-Calculus for high-school teachers
\\
1/8/2014 -- 30/8/2014& Undergraduate thesis reviewer&\textbf{Ministry of Education}\\
%In previous semesters I also tutored Algebra and Real Analysis.
7/1/2004 -- 30/1/2007& Military Service& \textbf{Paratrooper Reconnaissance Unit, IDF}, combatant \\
\end{longtable}
~\\
\subsection*{\sc Education}~
\begin{longtable}{p{.2\textwidth}p{.8\textwidth}}
\textbf{Doctorate}&
Ph.D Mathematics, Ben Gurion University of the Negev, Beer Sheva, Israel\\

&\textsc{Thesis}:\textit{ On Regularity and Approximative Regarding the Representation Zeta Functions of Groups}\\

 
&\textsc{Advisor}: Prof. Uri Onn\\
 
 
&\textsc{Submitted}: 2/10/2018 \quad\quad\textsc{Accepted}: 4/6/2019\\
\textbf{Master's}
&M.Sc Mathematics, Ben Gurion University of the Negev, Beer Sheva, Israel\\
&\textsc{Thesis}:\textit{ The Representation Zeta Function of the Special Linear Group of Certain Division Algebras over a Local Field}\\

&\textsc{Advisor}: Prof. Uri Onn
\\

\textbf{Undergraduate}&

B.Sc Mathematics, Ben Gurion University of the Negev, Beer Sheva, Israel, Graduated 2010% (GPA 87.9)\\
\\
%&B.A. Psychology and Neuroscience, Ben Gurion University of the Negev, Beer Sheva, Israel, Incomplete% (GPA 91.32)


\end{longtable}

\subsection*{\sc Published research papers}~\\
Citation numbers are based on Google Scholar and exclude self-citations.
\begin{longtable}{p{.05\textwidth}p{.95\textwidth}} 
3.&\fullcite{Shechter-Regular} \href{https://arxiv.org/abs/1709.01685}{arXiv\texttt{:1709.01685}} \quad(Citations: 1)\\

2.&\fullcite{CarnevaleShechterVoll} \href{https://arxiv.org/abs/1709.02717}{arXiv\texttt{:1709.02717}}\quad (Citations: 2)\\
1.
&\fullcite{Shechter-DivAlgs}
\href{https://arxiv.org/abs/1512.02448}{arXiv\texttt{:1512.02448}}\end{longtable}

%
%{\sf
%\subsubsection{\sf In preparation}
%
%Bounds on multiplicities of spherical spaces over finite fields - the general case
%
%
%\noindent Approximating the representation zeta functions of finite groups of Lie-type
%
%}

\subsection*{\sc Research Talks}
\subsubsection*{Invited and contributed presentations}~
\begin{longtable}{p{.1\textwidth}p{.9\textwidth}}
23/7/2019&  \textbf{Approximating the Representation Zeta Function of Finite Groups of Lie-type}, \underline{contributed talk} at the \textit{Advanced School on Representations of Pro-$p$ Groups}, ICMAT, Spain
\\
1/7/2017& \textbf{On Regular Characters of Classical Groups}, \underline{invited talk} at \textit{the Fourth International Workshop on Zeta Functions in Algebra and Geometry},  Bielefeld University, Germany
\\
18/6/2014 &\textbf{On The Representation Zeta Function for the Group $SL_1(D)$, for $D$ a Division Algebra Over a Local Field}
, \underline{poster} at the {\it Biccoca Workshop on Representation Theory}, UNIMIB, Italy
\\
23/9/2013&\textbf{On the Representation Zeta Function of the Special Linear Group of Division Algebras over Local Fields}, \underline{student lecture} at the workshop \textit{Zeta Functions of Groups and Related Subjects}, University of Padova, Italy\\
\end{longtable}
\subsubsection*{Selected seminar talks}~
\begin{longtable}{p{.1\textwidth}p{.9\textwidth}}
4/3/2019&\textbf{Approximating the Representation Zeta Function of Finite Groups of Lie-type}, Algebra Seminar, Haifa University, Israel\\
4/12/2018&\textbf{Approximating the Representation Zeta Function of Finite Groups of Lie-type}, Algebraic Geometry and Representation Theory Seminar, The Weizmann Institute of Science, Israel\\
22/6/2016&\textbf{On Regular Characters of Groups}, Algebraic Geometry and Number Theory Seminar, Ben-Gurion University\\
5/11/2014&\textbf{Representation Zeta Functions of Norm One Subgroups of a Local Division Algebra}, Algebra Seminar, Bar-Ilan University\end{longtable}
~\\
\subsection*{\sc Awards}
~
\begin{longtable}{p{.1\textwidth}p{.9\textwidth}}
8//5/2018 & Hillel Gauchman Scholarship for excellence in mathematical research (10,000 USD)
\end{longtable}
~\\
\subsection*{\sc Funding and Scholarships}~
\begin{longtable}{p{.22\textwidth}p{.78\textwidth}}
20/10/2018 -- Present day & Feinberg Graduate School Postdoctoral Fellowship at the Weizmann Insitute of Science\\
6/7/2014 -- 24/9/2014& Grant for summer visit at the Heinrich Heine University, D\"usseldorf, under the mentorship of Prof.~B.~Klopsch, NRW Scholarship program for students from Israel  \\
25/2/2014 -- 30/11/2019 & Negev Scholarship for PhD students, Ben-Gurion University- Zin Fellowship\\
10/7/2011-- 30/10/2011 & Summer Scholarship for outstanding applicants, Mathematics Department, Ben-Gurion University\\
1/11/2007 -- 1/10/2008 & Suzzane Zlotowski Scholarship for B.Sc students, Ben-Gurion University\\
\end{longtable}

\end{document}
