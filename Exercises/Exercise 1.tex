\documentclass[11pt, reqno,a4paper, twoside]{amsproc}
\usepackage{ifpdf}
\usepackage[english]{babel}
\usepackage{amsmath}
\usepackage{amsthm, amssymb,bm}
\usepackage[margin=1.5 cm]{geometry}
%usepackage{fullpage}
\usepackage{ucs}
\usepackage{tikz}
\usepackage{hyperref}
\usetikzlibrary{matrix}
\usepackage{mathrsfs}
\usepackage{eucal}
\hypersetup{
  colorlinks=true,
  citecolor=black,
  linkcolor=black,
  urlcolor=black,
  filecolor=red}
  
  
	
\usepackage{aliascnt}
\numberwithin{equation}{section}


\newtheorem{mainthm}{Theorem}\renewcommand{\themainthm}{\Roman{mainthm}}

\newtheorem{theo}{Theorem}[section]	
\newtheorem*{theo*}{Theorem}

\newaliascnt{lem}{theo}
\newtheorem{lem}[lem]{Lemma}
\aliascntresetthe{lem}

\newaliascnt{propo}{theo}
\newtheorem{propo}[propo]{Proposition}
\aliascntresetthe{propo}

\newaliascnt{corol}{theo}
\newtheorem{corol}[corol]{Corollary}
\aliascntresetthe{corol}

\newaliascnt{ques}{theo}
\newtheorem{ques}[ques]{Question}
\aliascntresetthe{ques}

\newaliascnt{conj}{theo}
\newtheorem{conj}[conj]{Conjecture}
\aliascntresetthe{conj}


\newaliascnt{exer}{theo}
\newtheorem{exer}[exer]{Exercise}
\aliascntresetthe{exer}

\newaliascnt{assumption}{theo}
\newtheorem{assumption}[assumption]{Assumption}
\aliascntresetthe{assumption}

\providecommand*{\mainthmautorefname}{Theorem}
\providecommand*{\theoautorefname}{Theorem}
\providecommand*{\propoautorefname}{Proposition}
\providecommand*{\lemautorefname}{Lemma}
\providecommand*{\corolautorefname}{Corollary}
\providecommand*{\quesautorefname}{Question}
\providecommand*{\assumptionautorefname}{Assumption}
\providecommand*{\conjautorefname}{Conjecture}
\providecommand*{\exerautorefname}{Exercise}


\theoremstyle{remark}

\newaliascnt{rem}{theo}
\newtheorem{rem}[rem]{Remark}
\aliascntresetthe{rem}
\providecommand*{\remautorefname}{Remark}
\newtheorem*{claim}{Claim}
\newaliascnt{exam}{theo}
\newtheorem{exam}[exam]{Example}
\aliascntresetthe{exam}
\providecommand*{\examautorefname}{Example}

\theoremstyle{definition}
\newaliascnt{defi}{theo}
\newtheorem{defi}[defi]{Definition}
\aliascntresetthe{defi}
\providecommand*{\defiautorefname}{Definition}
\newaliascnt{nota}{theo}
\newtheorem{nota}[nota]{Notation}
\aliascntresetthe{nota}
\providecommand*{\notaautorefname}{Notation}

\addto\extrasenglish{%
%  \renewcommand{\sectionautorefname}{Section}
  \renewcommand{\subsectionautorefname}{\S}
  \renewcommand{\subsubsectionautorefname}{\S\S}%
}





%%% Famous group schemes
\DeclareMathOperator{\GL}{GL}
\DeclareMathOperator{\UU}{U}
\DeclareMathOperator{\SL}{SL}
\DeclareMathOperator{\SU}{SU}
\DeclareMathOperator{\Sp}{Sp}
\DeclareMathOperator{\SO}{SO}
\DeclareMathOperator{\matr}{M}


%% Common Operators

\DeclareMathOperator{\Gal}{\bf Gal}
\DeclareMathOperator{\Lie}{Lie}
\DeclareMathOperator{\Stab}{Stab}
\DeclareMathOperator{\irr}{Irr}
\DeclareMathOperator{\End}{End}
\DeclareMathOperator{\aut}{Aut}
\newcommand{\Span}{\operatorname{Span}}
\DeclareMathOperator{\spec}{Spec}

\renewcommand{\ker}{\mathrm{Ker}}
\newcommand{\coker}{\mathrm{Coker}}
\renewcommand{\hom}{\mathrm{Hom}}
\newcommand{\res}{\mathrm{res}}
\newcommand{\ind}{\mathrm{Ind}}
\newcommand{\im}{\mathrm{Im}}
\newcommand{\Tr}{\mathrm{Tr}}
\newcommand{\rad}{\mathrm{Rad}}
\newcommand{\diag}{\mathrm{diag}}
\newcommand{\id}{\mathrm{Id}}
\newcommand{\Ql}{\underline{\dbQ_\ell}}

\newcommand{\Ad}{\mathrm{Ad}}
\newcommand{\Cen}{\ensuremath{\mathrm{C}}}

\newcommand{\supp}{\mathrm{Supp}}
\newcommand{\cl}[1]{\mathrm{cl}\left(#1\right)}
%\newcommand{\cl}[1]{\overline{#1}}

\renewcommand{\L}{\mcal{L}}
%% Famous Fields, Rings, Sets etc.

\renewcommand{\O}{\varnothing}
\newcommand{\dbN}{\mathbb N}
\newcommand{\dbZ}{\mathbb Z}
\newcommand{\dbF}{\mathbb F}
\newcommand{\dbQ}{\mathbb Q}
\newcommand{\dbR}{\mathbb R}
\newcommand{\dbC}{\mathbb C}
\newcommand{\dbA}{\mathbb A}
\newcommand{\dbP}{\mathbb P}

\newcommand{\kk}{k}
\newcommand{\B}{\mcal{B}}

\newcommand{\frob}{\sigma}

%% Common Unary Functions
\newcommand{\gen}[1]{\langle{#1}\rangle}
\newcommand{\set}[1]{\left\{{#1}\right\}}
\newcommand{\norm}[1]{\left\|#1\right\|}
\newcommand{\abs}[1]{\left|#1\right|}
\newcommand{\inner}[1]{\left(#1\right)}


%% Shorter font named
\newcommand{\mcal}{\mathcal}
\newcommand{\mbf}{\mathbf}
\newcommand{\mfr}{\mathfrak}
\newcommand{\msf}{\mathsf}

\newcommand{\widebar}{\overline}
\renewcommand{\tilde}{\widetilde}

%% 
\renewcommand{\thesubsection}{Exercise~\arabic{subsection}}
\title{Algebraic Geometry 2\\Exercise sheet 1}
\begin{document}\maketitle

Solve the following exercises. Exercises marked with $*$ are optional.
\subsection{} Let $X$ be a topological space, $p\in X$ a point and $A$ an abalian group. Define, for $U\subseteq X$ open
\[i_{p,A}(U)=\begin{cases}
A&\text{if }p\in U\\0&\text{otherwise}.
\end{cases}\]
\begin{enumerate}
	\item 
	Show that $U\mapsto i_{p,A}(U)$ defines a sheaf of abelian groups on $X$, with the restriction maps defined, for $V\subseteq U$ open, by $\res_{U,V}=\id_A$ if $p\in V$ and $\res_{U,V}=0$ if $p\notin V$. 
	\item 
	* Is it possible to replace the restriction maps $\res_{U,V}$ with different in order to obtain a sheaf structure on $i_{p,A}$?
\end{enumerate}

\subsection{}\label{question:holom} Let $X=\dbC\setminus{\set{0}}$ with the standard topology. Given $U\subseteq X$, let
\begin{align*}
	\mcal{F}(U)&=\set{\varphi:U\to \dbC\mid\varphi \text{  is holomorphic}}\intertext{ and }
	\mcal{G}(U)&=\set{\varphi:U\to \dbC\mid\varphi\text{ is holomorphic and non-vanishing on }U}.
\end{align*}
\begin{enumerate}
	\item Show that $\mcal{F}$ and $\mcal{G}$ are sheaves of abelian groups on $X$, where the group operation on $\mcal{F}(U)$ is addition of functions, and on $\mcal{G}(U)$ is multiplication.\footnote{Recall that a function $\varphi:U\to\dbC$ is holomorphic iff it is analytic on $U$, i.e. if for any $x\in U$ there is a neighbourhood $x\in V\subseteq U$ such that $\varphi\mid_V$ is given by a power series. If you are not comfortable with either of these terms, you may replace the term `holomorphic' with `continuous'  at each occurrence and solve the analogous question for continuous functions.}
	\item Define a morphism $\varphi:\mcal{F}\to\mcal{G}$ by setting $\varphi_U:\mcal{F}(U)\to\mcal{G}(U)$ to be $\varphi_U(f)(z)=e^{f(z)}$. 
	\begin{enumerate}
		\item Show that $\varphi$ is a morphism of sheaves.
		\item Describe $\ker(\varphi)$; show that it is isomorphic to the sheaf of locally constant functions with values in $\dbZ$.
		\item Let $U_1=\dbC\setminus\dbR_{\ge 0}$ and $U_2=\dbC\setminus\dbR_{\le 0}$.  Show that $\varphi_{U_1}$ and $\varphi_{U_2}$ are surjective. Is $\varphi_{(U_1\cup U_2)}$ surjective as well?
		\item Deduce that the assignment $U\mapsto \im(\varphi_U)$ defines a presheaf which is not a sheaf.
	\end{enumerate}

\end{enumerate}
\subsection{} Let $X$ be a topological space and let $\mcal{F},\mcal{G}$ be presheaves of abelian groups on $X$ with $\varphi:\mcal{F}\to\mcal{G}$ a morphism of presheaves. 
	\begin{enumerate}
		\item Given $p\in X$, describe the induced morphism  $\varphi_p:\mcal{F}_p\to\mcal{G}_p$ on the stalks of $\mcal{F}$ and $\mcal{G}$ at $p$. Show that it is a group homomorphism. 
		\item Show that $\varphi$ is injective (i.e. $\ker(\varphi)$ is the constant zero sheaf) if and only if $\varphi_p$ is injective for all $p\in X$. 
		\item Show that in the setting of \ref{question:holom}.(2), the map $\varphi_p:\mcal{F}_p\to\mcal{G}_p$ is surjective for all $p\in X$. 
	\end{enumerate}

\subsection{} Let $\pi:M\to X$ be a continuous surjective map of topological spaces. 
\begin{enumerate}
	\item Given $U\subseteq X$ open, let $\mcal{S}(U)=\set{s:U\to M\mid \pi\circ s=\id_U}$. Show that $\mcal{S}$ is a sheaf of sets over $X$. 
	\item (*\footnote{This exercise is optional because acquaintance of vector bundles over $C^\infty$ manifold is not a prerequisite of this course. If you \textit{are} acquainted with this notion, or are willing to look them up, it is advisable to submit this exercise in any case.}) Assume $X$ is a $C^\infty$ manifold and $\pi:M\to X$ is a vector bundle. Let \[\mcal{S}^{\rm diff}(U)=\set{s:U\to M\mid s\text{ is differentiable and }\pi\circ s=\id_U}.\]
	Show that $\mcal{S}^{\rm diff}$ is a sheaf of abelian groups on $X$. 
\end{enumerate}
\end{document}
