\documentclass[12pt, reqno,a4paper, twoside]{amsproc}
\usepackage{ifpdf}
\usepackage[english]{babel}
\usepackage{amsmath}
\usepackage{amsthm, amssymb,bm}
\usepackage{geometry}
\usepackage{fullpage}
\usepackage{ucs}
\usepackage{tikz}
\usepackage{hyperref}
\usetikzlibrary{matrix}
\usepackage{mathrsfs}
\usepackage{eucal}
\hypersetup{
  colorlinks=true,
  citecolor=black,
  linkcolor=black,
  urlcolor=black,
  filecolor=red}
  
  
	
\usepackage{aliascnt}
\numberwithin{equation}{section}


\newtheorem{mainthm}{Theorem}\renewcommand{\themainthm}{\Roman{mainthm}}

\newtheorem{theo}{Theorem}[section]	
\newtheorem*{theo*}{Theorem}

\newaliascnt{lem}{theo}
\newtheorem{lem}[lem]{Lemma}
\aliascntresetthe{lem}

\newaliascnt{propo}{theo}
\newtheorem{propo}[propo]{Proposition}
\aliascntresetthe{propo}

\newaliascnt{corol}{theo}
\newtheorem{corol}[corol]{Corollary}
\aliascntresetthe{corol}

\newaliascnt{ques}{theo}
\newtheorem{ques}[ques]{Question}
\aliascntresetthe{ques}

\newaliascnt{conj}{theo}
\newtheorem{conj}[conj]{Conjecture}
\aliascntresetthe{conj}


\newaliascnt{exer}{theo}
\newtheorem{exer}[exer]{Exercise}
\aliascntresetthe{exer}

\newaliascnt{assumption}{theo}
\newtheorem{assumption}[assumption]{Assumption}
\aliascntresetthe{assumption}

\providecommand*{\mainthmautorefname}{Theorem}
\providecommand*{\theoautorefname}{Theorem}
\providecommand*{\propoautorefname}{Proposition}
\providecommand*{\lemautorefname}{Lemma}
\providecommand*{\corolautorefname}{Corollary}
\providecommand*{\quesautorefname}{Question}
\providecommand*{\assumptionautorefname}{Assumption}
\providecommand*{\conjautorefname}{Conjecture}
\providecommand*{\exerautorefname}{Exercise}


\theoremstyle{remark}

\newaliascnt{rem}{theo}
\newtheorem{rem}[rem]{Remark}
\aliascntresetthe{rem}
\providecommand*{\remautorefname}{Remark}
\newtheorem*{claim}{Claim}
\newaliascnt{exam}{theo}
\newtheorem{exam}[exam]{Example}
\aliascntresetthe{exam}
\providecommand*{\examautorefname}{Example}

\theoremstyle{definition}
\newaliascnt{defi}{theo}
\newtheorem{defi}[defi]{Definition}
\aliascntresetthe{defi}
\providecommand*{\defiautorefname}{Definition}
\newaliascnt{nota}{theo}
\newtheorem{nota}[nota]{Notation}
\aliascntresetthe{nota}
\providecommand*{\notaautorefname}{Notation}

\addto\extrasenglish{%
%  \renewcommand{\sectionautorefname}{Section}
  \renewcommand{\subsectionautorefname}{\S}
  \renewcommand{\subsubsectionautorefname}{\S\S}%
}





%%% Famous group schemes
\DeclareMathOperator{\GL}{GL}
\DeclareMathOperator{\UU}{U}
\DeclareMathOperator{\SL}{SL}
\DeclareMathOperator{\SU}{SU}
\DeclareMathOperator{\Sp}{Sp}
\DeclareMathOperator{\SO}{SO}
\DeclareMathOperator{\matr}{M}


%% Common Operators

\DeclareMathOperator{\Gal}{\bf Gal}
\DeclareMathOperator{\Lie}{Lie}
\DeclareMathOperator{\Stab}{Stab}
\DeclareMathOperator{\irr}{Irr}
\DeclareMathOperator{\End}{End}
\DeclareMathOperator{\aut}{Aut}
\newcommand{\Span}{\operatorname{Span}}
\DeclareMathOperator{\spec}{Spec}

\renewcommand{\ker}{\mathrm{Ker}}
\newcommand{\coker}{\mathrm{Coker}}
\renewcommand{\hom}{\mathrm{Hom}}
\newcommand{\res}{\mathrm{Res}}
\newcommand{\ind}{\mathrm{Ind}}
\newcommand{\im}{\mathrm{Im}}
\newcommand{\Tr}{\mathrm{Tr}}
\newcommand{\rad}{\mathrm{Rad}}
\newcommand{\diag}{\mathrm{diag}}
\newcommand{\id}{\mathbf{1}}
\newcommand{\Ql}{\underline{\dbQ_\ell}}

\newcommand{\Ad}{\mathrm{Ad}}
\newcommand{\Cen}{\ensuremath{\mathrm{C}}}

\newcommand{\supp}{\mathrm{Supp}}
\newcommand{\cl}[1]{\mathrm{cl}\left(#1\right)}
%\newcommand{\cl}[1]{\overline{#1}}

\renewcommand{\L}{\mcal{L}}
%% Famous Fields, Rings, Sets etc.

\renewcommand{\O}{\varnothing}
\newcommand{\dbN}{\mathbb N}
\newcommand{\dbZ}{\mathbb Z}
\newcommand{\dbF}{\mathbb F}
\newcommand{\dbQ}{\mathbb Q}
\newcommand{\dbR}{\mathbb R}
\newcommand{\dbC}{\mathbb C}
\newcommand{\dbA}{\mathbb A}
\newcommand{\dbP}{\mathbb P}

\newcommand{\kk}{k}
\newcommand{\B}{\mcal{B}}

\newcommand{\frob}{\sigma}

%% Common Unary Functions
\newcommand{\gen}[1]{\langle{#1}\rangle}
\newcommand{\set}[1]{\left\{{#1}\right\}}
\newcommand{\norm}[1]{\left\|#1\right\|}
\newcommand{\abs}[1]{\left|#1\right|}
\newcommand{\inner}[1]{\left(#1\right)}


%% Shorter font named
\newcommand{\mcal}{\mathcal}
\newcommand{\mbf}{\mathbf}
\newcommand{\mfr}{\mathfrak}
\newcommand{\msf}{\mathsf}

\newcommand{\widebar}{\overline}
\renewcommand{\tilde}{\widetilde}

\newcommand{\loc}{\mathrm{loc}}

%% 

\newenvironment{sol}{\sc Solution. \rm}{\hfill \qedsymbol\bigskip}

\title{Generalized functions\\Tutorial notes}
\begin{document}\maketitle
\part*{Tutorial 3}\setcounter{section}{3}
\subsection{Complements from previous tutorial}
At the beginning of the tutorial, we discussed two subjects which we did not manage to cover in the previous tutorial. These subjects were:
\begin{itemize}
	\item distributions on $\dbR$ with support $\set{0}$; and
	\item convolution of distributions. 
\end{itemize}
For the full discussion, see Tutorial 2 notes.
\subsection{Topological vector spaces}
\begin{defi}
	A topological vector space $V$ is a vector space over a topological field $F$ (i.e. a field with a topology under which the field operations are continuous) such that $+:V\times V\to V$ and $\cdot:F\times V\to V$ are continuous. 
\end{defi}
\begin{exer}
	Prove that a topological vector space is Hausdorff iff $\set{0}$ is a closed set.
\end{exer}
\begin{sol}
	\begin{itemize}
		\item[$\Leftarrow$] 
		Clear; in a Hausdorff space all points are closed.
		\item[$\Rightarrow$] 
		Consider the set $\Delta=\set{(x,x):x\in V}$. Then $\Delta=f^{-1}(\set{0})$, for $f(u,v)=u-v$, which is closed whenever $\set{0}$ is closed, since $f$ is continuous. Let $u,v\in V$ be two distinct points. Then $(u,v)\notin \Delta$ and hence there exists $(u,v)\in  W\subseteq V\times V$ open which is disjoint from $\Delta$. Since the topology on $V\times V$ is generated by boxes, i.e. sets of the form $U_1\times U_2$ For $U_1,U_2\subseteq V$ open, we have that $(u,v)\in U_1\times U_2\subseteq W$. Hausdorffness follows, since $U_1\times U_2\cap \Delta=\O$ is equivalent to $U_1\cap U_2=\O$. 
	\end{itemize}
\end{sol}
\begin{defi}[Local convexity etc]
	Let $V$ be a topological vector space over $F=\dbR$ or $\dbC$.
	\begin{enumerate}
		\item A set $A\subseteq V$ is \emph{convex} is $\lambda A+(1-\lambda)A\subseteq A$ for all $\lambda\in[0,1]$. 
		\item $V$ is said to be \emph{locally convex} if its topology can be generated by open convex sets.
		\item  A set $W\subseteq V$ is said to be \emph{balanced} if $\lambda W\subseteq W$ whenever $\abs{\lambda}\le 1$ and $\lambda\in F$. 
		\item  Given a (balanced open convex) set $C\ni 0$, one defines
		\[N_C(x):=\inf\set{\alpha\in \dbR_{\ge 0}:x\in \alpha C}\quad(x\in V).\]
		\item  A set $C$ is said to be \emph{absorbent} if $N_C(x)<\infty$ for all $x\in V$.
	\end{enumerate}
\end{defi}

\begin{exer}
	Find a topological vector space $V$ which is not locally convex.
\end{exer}
\begin{sol}
	Note that if $V$ is normed, or, more generally equipped with a translation invariant metric $d$ such that $d(\lambda x,\lambda y)\le \abs{\lambda}d(x,y)$ for all $x,y\in V$ and $\abs{\lambda}<1$, then any open ball around $0$ in $V$ is convex and so $V$ is locally convex.
	
	Let $V=\ell^{1/2}(\dbR)=\set{(x_i)_{i=1}^\infty:x_i\in \dbR,\:\sum\sqrt{\abs{x_i}}<\infty}, $ equipped with the topology induced from the metric $d((x_i),(y_i))=\sum_i\sqrt{\abs{x_i-y_i}}$. Consider the open ball $B_1(0)$ of radius $1$ around $0$. Note that $B_1(0)$ is \textit{not} convex, e.g.\ for $x=(1/2,0,0,\ldots), y=(0,1/2,0,0,\ldots)$
	we have that $d(x,0)=d(y,0)=1/\sqrt{2}<1$, but $d(\frac{1}{2}x+\frac{1}{2}y,0)=\sqrt{1/4}+\sqrt{1/4}=1$. 
	
	Assume towards a contradiction that $V$\textit{ is} is locally convex. Since $B_1(0)$ is open, it contains a convex open subset $0\in C\subseteq B_1(0)$, which, in turn, contains a smaller open ball $B_\epsilon(0)$ around zero. Since $C$ is convex, it follows that any convex combination of elements of $B_\epsilon(0)$ must be included in $C$. On the other hand, if we take $x_n=(x_n^i)_{i=1}^\infty$, defined by $x_n^i=\epsilon^2$ if $i=n$ and $0$ otherwise, then, for any $n\in \dbN$,
	\[d(\sum_{i=1}^n\frac{1}{n}x_n^i,0)=\sum_{i=1}^n\frac{\epsilon}{\sqrt{n}}=\sqrt{n}\epsilon,\]
	which tends to infinity as $n$ grows, and , in particular, eventually escapes the ball $B_1(0)$.
\end{sol}

\begin{defi}[Seminorm]
	A \textit{seminorm} on a topological vector space is a function $\eta:V\to \dbR$ which satisfyies the triangle inequality, homogeneity and non-negativity axioms, but such that $\eta(v)=0$ may be possible for $v\ne 0$.
\end{defi}
\begin{exer}
	Let $C$ be an open convex neighborhood of $0$ in a tvs $V$ over $\dbR$. 
	\begin{enumerate}
		\item Show that $C$ is absorbent. 
		\item Show that if $C$ is further assumed to be balanced than $N_C$ is a seminorm.
	\end{enumerate}
\end{exer}
\begin{sol}
	\begin{enumerate}
		\item Let $v\in V$ be arbitrary. Consider the set $\tilde{C}=\set{(\lambda,u):\dbR\times V:\lambda u\in C}$. This is just the preimage of $C$ under scalar multiplication, and hence is open in $F\times V$. Also, it clearly contains $(0,v)$ (and, more generally, $(0,u)$ for all $u\in V$). In particular, there exist $0\in U_1\subseteq \dbR$ and $v\in U_2\subseteq V$ open such that $(0,v)\in U_1\times U_2\subseteq \tilde{C}$. Since $U_1$ is open in $\dbR$ it contains non-zero elements, and there exists $\lambda\ne 0$ such that $(\lambda,v)\in \tilde{C}$, and so $\lambda v\in C$ and $v\in (\lambda^{-1})C$, as wanted.
		
		\item Home exercise.%Non-negativity is obvious; Homogeneity follows easily from balancedness; finally, the triangle inequality follows from convexity of $C$: indeed if $x,y\in V$ and $\frac{1}{\beta}x,\frac{1}{\gamma}y\in C$, then $\frac{\beta}{\beta+\gamma}\frac{1}{\beta}x+\frac{\gamma}{\beta+\gamma}\frac{1}{\gamma} y\in C$ by convexity, implying that $\beta+\gamma\ge N_C(x+y)$. The lemma follows by replacing $\gamma$ and $\beta$ by their infima.
	\end{enumerate}
\end{sol}
%
%\begin{exer}
%	Find a non-normable tvs.
%\end{exer}
%\begin{sol}
%	This will appear in the home exercises, the big question is- given a tvs, how do you known that the topology isn't normable? Or put otherwise, can you reconstruct the norm from the open sets? The answer is, essentially, yes.
%\end{sol}

\begin{theo}[Hahn-Banach]
	Let $V$ be a \emph{normed} vector space and $W\subseteq V$ a subspace with $f:W\to \dbR$ a bounded linear functional (i.e. such that $\norm{f}=\sup_{x\in W,\: \norm{x}=1}\norm{f(x)}<\infty$ for some $C>0$ and for all $x\in W$). Then there exists a linear functional $\tilde{f}:V\to \dbR$ such that $\tilde{f}\mid_W=f$ and $\norm{\tilde{f}}=\norm{f}$. 
\end{theo}
\begin{exer}
	Let $V$ be a locally convex topological vector space and $W$ a closed linear functional. Show that any continuous linear functional $f:W\to \dbR$ can be extended to $V$. 
\end{exer}
\begin{sol}\renewcommand{\bar}{\overline}
	Let $f:W\to \dbR$ be a continuous linear functional. By continuity and the definition of the induced topology, $f^{-1}(-1,1)=A\cap W$ for some open set $A$. By local convexity of $V$, we have that $A$ contains an open convex set $C\ni 0$, which we can further assume to be balanced, by the home exercise. One easily verifies that $\abs{f(x)}\le N_C(x)$ for all $x\in W$; indeed $\frac{x}{N_C(x)+\epsilon}\in C\subseteq f^{-1}(-1,1)$, for all $\epsilon>0$. Note that $U:=\ker(N_C)=\set{x\in V:N_C(x)=0}$ is a closed\footnote{Verify that you see why $U$ is closed.} linear subspace of $V$; indeed, for any $x,y\in U$, $\lambda\in F$, $N_C(\lambda x)=\abs{\lambda}N_C(x)=0$ and $0\le N_C(x+y)\le N_C(x)+N_c(y)=0$. Furthermore, $f(U)=0$, since $\abs{f(x)}\le N_C(x)=0$ for all $x\in U$. In particular, $f$ reduces to a linear functional on $\bar{W}=W/U\subseteq V/U=\bar{V}$. Finally, we note that $N_C$ reduces to a  \emph{norm} on $\bar{V}$, and hence we can apply Hahn-Banach to extend the map induced from $f$ on $\bar{W}$ to $\bar{V}$, and then pull back to an extension $\tilde{f}$ of $f$ to $V$.
\end{sol}

\begin{corol}
	Given a locally convex vector space $V$ with a closed subspace $W$, the restriction maps $V^\sharp\to W^\sharp$ and $V^*\to W^*$ are surjective (here $V^\sharp$ is the abstract dual $\hom(V,\dbR)$, consisting of \emph{all} linear maps $V\to\dbR$).
\end{corol}

\subsection{Complete and sequentially complete topological vector spaces}

\begin{defi}
	Let $V$ be a topological vector space. 
	\begin{enumerate}
		\item A sequence $\set{v_n}_{n=1}^\infty$ in $V$ is Cauchy if for every neighbourhood $U$ of $0$, there exists $n_0\in \dbN$ such that $v_m-v_n\in U$ for any $m,n\in \dbN$.%; Similarly, a net $\set{v_\alpha}_{\alpha\in I}$ is Cauchy if for any such $U$ there exists $\alpha_0\in I$ such that $v_\alpha-v_\beta\in U$ whenever $\alpha,\beta>\alpha_0$. 
		\item A sequence $\set{v_n}_{n=1}^\infty$ is said converge to $v\in V$ if for every $0\in U$ open, there exists $n_0\in \dbN$ such that $v_n-v\in U$ for all $n>n_0$. 
		\item $V$ is called \emph{sequentially complete} if all Cauchy sequences converge to some limit in $V$;
		\item  $V$ is said to be complete if for every $\phi:V\to W$ which maps $V$ homeomorphically onto $\phi(V)$, the set $\phi(V)$ is closed in $W$.		
	\end{enumerate}
\end{defi}
\begin{exer}
	Find a topological vector space which is complete sequentially complete but not complete.
\end{exer}
\begin{sol}
	Let $V=\dbR^\dbR=\set{f:\dbR\to\dbR}$ have the product topology ($\dbR$ is endowed with the standard topology), and let $U=\set{f:\dbR\to\dbR\mid \abs{\set{x:f(x)\ne 0}}\le \aleph_0}$. Given a Cauchy sequence $(f_n)\in U$, since coordinate projections are continuous in the product topology, the sequence $f_n(x)$ is Cauchy in $\dbR$ for all $x\in \dbR$. We may define $f(x)=\lim_{\to\infty}f_n(x)$, and this is clearly an element of $U$, since it can have at most countably many non-zero values. Also, using the definition of the product topology, one easily verifies that $f_n$ converges to $f$ in $V$. Thus $U$ is sequentially complete. 
	
%	Proving that $U$ is not complete can be done using the Baire category theorem (a complete vector space cannot have a countable basis)
	Also, from the definition of the product topology, one has that $U$ is dense in $V$, and doe not equal it, so it is not complete. 
\end{sol}

\begin{rem}
	Another important example of a sequentially complete but not complete space is the image of $C_c^{\infty}(\dbR)$ in $C^{-\infty}(\dbR)$, under the map $f\mapsto \xi_f$ (where $\gen{\xi_f,g}=\int_\dbR f(x)g(x)dx$), with respect to discrete topology on $C^{-\infty}(\dbR)$. The proof of this fact, which relies on the Banach-Steinhaus Theorem, will appear in the next tutorial.
\end{rem}

We also have the following universal description of the completion of $V$:
\begin{exer}
	Let $V$ be a topological vector space and $\iota:V\to \bar{V}$ be an embedding into another tvs. Prove that the following are equivalent:
	\begin{enumerate}
		\item $\iota(V)\simeq V$ and $\cl{\iota(V)}=\bar{V}$; and
		\item For every complete space $W$ and $f:V\to W$ there exists a unique map $\varphi_W:\bar{V}\to W$ such that $f=\varphi_W\circ\iota$.
	\end{enumerate}
\end{exer}
%\begin{sol}
%	
%\begin{enumerate}
%	\item[(1)$\Rightarrow$(2)] Let $W$ be a space and $f:V\to W$. Since $\iota$ is an embedding, we may define a map $\varphi:\iota(V)\to W$ by $\varphi(\iota(v))=f(v)$, which is clearly well-defined and continuous. As $\iota(V)$ is dense in $\bar{V}$ and $W$ is complete, the map $\varphi$ extends uniquely to a map $\varphi_W:\bar{V}\to W$ (e.g. by taking the values of $\varphi$ on nets in $V$), such that $\varphi_W\circ\iota=f$. 
%	\item[(2)$\Rightarrow$(1)] Put $W=\cl(\iota(V))\subseteq \bar{V}$, which is a closed subset of a complete space ($\bar{V}$) and hence is complete. Let $f:V\to W$ be the restriction (on the range) of $\iota$, and $\varphi_W:\bar{V}\to W$ be the corresponding extension given by (2), so that $f=\varphi_W\circ\iota=\varphi_W\circ f$. 
%	
%	Let $j:W\to \bar{V}$ denote the inclusion map and consider the map $j\circ \varphi_W:\bar{V}\to\bar{V}$. %. Then $\varphi_W\circ j\circ \iota(v)=\varphi_W(\iota(v))=\iota(v)$, and hence, by the uniqueness in (2), we must have $(\varphi_W\circ j)=\id_W$.
%	Note that $j\circ \varphi_W\circ\iota= j\circ f=\iota$. By applying the assumption to the map $\iota:V\to \bar{V}$, ans since $\id_{\bar{V}}$ is another map satisfying the same equality, we deduce that $j\circ\varphi_W=\id_V$ and, in particular, that $j$ is \textit{surjective}. Thus $\bar{V}=W$. 
%\end{enumerate}	
%\end{sol}
Finally, using Cauchy filters or Cauchy nets, one can construct the completion of a topological vector space explicitly, and prove:
\begin{exer}
	The completion of a tvs $V$ exists and is unique up to unique isomorphism.
\end{exer}

\end{document}
\subsection{Fr\'chet spaces}
\begin{defi}
	\begin{enumerate}
		\item  Metrizable space;
		\item  $S_1$ or first countable, if every point has a countable basis;
		\item Fr\'echet =locally convex complete metrizable tvs.
	\end{enumerate}
\end{defi}
\begin{exam} $C^{\infty}(K)$ is Fr\`echet, but not normed (with the norm of uniform convergence of all derivatives). 
\end{exam}
\begin{theo}[Banach-Steinhaus for Fr\`echet spaces] Let $X$ be a Fr\`echet space and $Y$ a normed space, and $F$ a family of continuous linear functionals. Assume $\sup_{f\in F}\norm{f(x)}<\infty$ for any $x\in X$. Then the family $F$ is equicontinuous, i.e. for any $\epsilon>0$ there exists $\delta>0$ such that $\norm{f(x_1)-f(x_2)}<\epsilon$ for any $f\in F$ and $x_1,x_2\in X$ with $d(x_1,x_2)<\delta$. 
\end{theo}


\begin{exer}
	TFAE for a lctvs:
	\begin{enumerate}
		\item metrizability;\item  first countable; \item countable collection of seminorms
	\end{enumerate}
\end{exer}
In particular: complete+either of the above gives Fr\`echet.

\begin{exer}
	Show $C^\infty(\mathbb {S}^1)\simeq \mcal{S}(\dbZ)$, where $\mcal{S}(\dbZ)=\set{(x_n)_{n\in\dbZ}:\sup\abs{x_n}n^\alpha<\infty\text{ for all }\alpha\in\dbN}$ is the space of rapidly decaying sequences.
\end{exer}
\begin{sol}
%	Both spaces are Fr\'echet ($C^\infty(\mathbb S^1)$ is a special case of $C^\infty(K)$ and $\mcal{S}(\dbZ)$ )
\newcommand{\F}{\mcal{F}}\newcommand{\dbS}{\mathbb S}
	Let $\F:C^\infty(\dbS^1)\to \mcal{S}(\dbZ)$ denote the Fourier transform, given by $\F(f)=\left(\frac{1}{2\pi}\int_{0}^{2\pi}f(x)e^{- n  xi}dx\right)_{n\in\dbZ}$, with inverse $\F^{-1}((c_n)_{n\in\dbZ})=\sum_{n=-\infty}^\infty c_ne^{inx}$. Let $\eta_j(f):=\sup_{x\in \dbS^1}\sup\abs{f^{(j)}(x)}$ and $\nu_j((c_n)_n)=\sup_{n\in\dbZ}\abs{n^{j}c_n}$.
	
	Given $f\in C^\infty(\dbS^1)$ with $\F(f)=(c_n)_{n\in\dbZ}$, and $k\in\dbZ$, we have that
	\begin{multline*}
	\abs{k^j c_k}=\abs{\gen{f,e^{-ikx}}}=\abs{\sum_{n\in\dbZ}\frac{1}{2\pi}\int_0^{2\pi}c_nn^j e^{i(n-k)x}dx}\\
	\le\frac{1}{2\pi}\int_0^{2\pi}\abs{\sum_{n\in\dbZ}c_nn^j e^{i(n-k) x}}dx=\frac{1}{2\pi}\int_0^{2\pi}\abs{f^{(j)}e^{-ikx}}dx\\
	\le \frac{1}{2\pi}\int_0^{2\pi}\sup_{x\in \dbS^1}\abs{f^{(j)}(x)}dx=\eta_j(f)
	\end{multline*}
	
	In particular, we have that $\nu_j(\mcal{F})\le \eta_j(f)$.
	On the other hand, we have that
	\begin{multline*}
	\eta_j(f)=\sup_{x\in\dbS^1}\abs{f^{(j)}(x)}=\sup_{x\in\dbS^1}\sum_{n\in\dbZ} c_n(ni)^j e^{inx}\le \\\sum_{n\in\dbZ\setminus{\set{0}}}\frac{1}{n^2}\abs{c_nn^{j+2}}\le\sup_{k\in\dbZ}\abs{k^{j+2}c_k}\sum_{n\in\dbZ\setminus{\set{0}}} \frac{1}{n^2}=\frac{2\pi^2}{6}\nu_{j+2}((c_n).
	\end{multline*}
%Let $B_\epsilon^{\eta_j}(0))=\set{f\in C^\infty(\dbS^1):\eta_j(f)<1}$. Then 
%	\[\F(B_\epsilon^{\eta_j}(0))=\set{(c_n)_{n\in\dbZ}:\abs{\F^{-1}((c_n n^{-j})_n)(x)}=\abs{\sum_n c_nn^{-j}e^{inx}}<\infty\text{ for all x}}.\]
%	Indeed, the $n$-th Fourier coefficient of $f^(j)$ is $c_n n{^-j}$ (follows from change of variable).
	\end{sol}
\end{document}


\item $C^{-\infty}(\dbR)$ is weakly sequentially complete [Wait for completions subsection, Include Banach-Steinhaus]
\begin{proof}
	Recall the Banach-Steinhaus Theorem (for normed spaces; reiterate for Fr\`echet spaces): 
	\begin{theo}[Banach-Steinhaus]
		Let $X$ be Banach space and $Y$ a normed vector space, and let $F\subseteq \hom(X,Y)$ be a family of continuous linear transformations. Assume that for any $x\in X$, $\sup_{f\in F}\norm{f(x)}_Y<\infty$ (i.e. $F$ is pointwise bounded). Then $F$ is uniformly bounded, i.e. $\sup_{f\in F, \norm{x}=1} \norm{f(x)}=\sup_{f\in F}\norm{f}_{\hom(X,Y)}<\infty$. 
	\end{theo}
	
	Let $(\xi_n)$ be a weakly Cauchy sequence in $(C_c^{\infty}(\dbR))^*$. We have a natural candidate for the limit of $(\xi_n)$, which is defined by
	\[\gen{\xi,f}:=\lim_{n\to\infty}\gen{\xi_n,f},\]
	which exists by definition of weakly Cauchy. $\xi$ is clearly linear, but not obviously continuous.
	
	Let $K\subseteq\dbR$ be a compact set. We consider the restrictions of the $\xi_n$'s and $\xi$ to $C^\infty(K)$. More specifically, we consider $F=\set{\xi_n\mid_{C^\infty(K)},\xi\mid_{C^\infty(K)}}\subseteq\hom(C^\infty(K),\dbR)$. By weak convergence, $F$ is pointwise bounded, and hence, by Banach-Steinhaus, it is uniformly bounded, i.e. $c=\sup_{\nu\in F}\norm{\nu}<\infty$. Let $f_n$ be a sequence in $C^\infty(K)$, converging to $f$ (note that $C^\infty(K)$ is complete with respect to the topology of uniform convergence of all derivatives, so $f\in C^\infty(K)$). Then $\norm{f_n-f}_{C^\infty(K)}\xrightarrow{n\to\infty}0$, and hence \[\abs{\gen{\xi,f-f_n}}\le \sup_{\nu\in F}\abs{\gen{\nu,f-f_n}}\le \sup_{\nu\in F}\norm{\nu}\cdot \norm{f-f_n}\xrightarrow{n\to\infty}0.\]\end{proof}