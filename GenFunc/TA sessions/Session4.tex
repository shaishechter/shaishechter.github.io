\documentclass[12pt, reqno,a4paper, twoside]{amsproc}
\usepackage{ifpdf}
\usepackage[english]{babel}
\usepackage{amsmath}
\usepackage{amsthm, amssymb,bm}
\usepackage{geometry}
\usepackage{fullpage}
\usepackage{ucs}
\usepackage{tikz}
\usepackage{hyperref}
\usetikzlibrary{matrix}
\usepackage{mathrsfs}
\usepackage{eucal}
\hypersetup{
  colorlinks=true,
  citecolor=black,
  linkcolor=black,
  urlcolor=black,
  filecolor=red}
  
  
	
\usepackage{aliascnt}
\numberwithin{equation}{section}


\newtheorem{mainthm}{Theorem}\renewcommand{\themainthm}{\Roman{mainthm}}

\newtheorem{theo}{Theorem}[section]	
\newtheorem*{theo*}{Theorem}

\newaliascnt{lem}{theo}
\newtheorem{lem}[lem]{Lemma}
\aliascntresetthe{lem}

\newaliascnt{propo}{theo}
\newtheorem{propo}[propo]{Proposition}
\aliascntresetthe{propo}

\newaliascnt{corol}{theo}
\newtheorem{corol}[corol]{Corollary}
\aliascntresetthe{corol}

\newaliascnt{ques}{theo}
\newtheorem{ques}[ques]{Question}
\aliascntresetthe{ques}

\newaliascnt{conj}{theo}
\newtheorem{conj}[conj]{Conjecture}
\aliascntresetthe{conj}


\newaliascnt{exer}{theo}
\newtheorem{exer}[exer]{Exercise}
\aliascntresetthe{exer}

\newaliascnt{assumption}{theo}
\newtheorem{assumption}[assumption]{Assumption}
\aliascntresetthe{assumption}

\providecommand*{\mainthmautorefname}{Theorem}
\providecommand*{\theoautorefname}{Theorem}
\providecommand*{\propoautorefname}{Proposition}
\providecommand*{\lemautorefname}{Lemma}
\providecommand*{\corolautorefname}{Corollary}
\providecommand*{\quesautorefname}{Question}
\providecommand*{\assumptionautorefname}{Assumption}
\providecommand*{\conjautorefname}{Conjecture}
\providecommand*{\exerautorefname}{Exercise}


\theoremstyle{remark}

\newaliascnt{rem}{theo}
\newtheorem{rem}[rem]{Remark}
\aliascntresetthe{rem}
\providecommand*{\remautorefname}{Remark}
\newtheorem*{claim}{Claim}
\newaliascnt{exam}{theo}
\newtheorem{exam}[exam]{Example}
\aliascntresetthe{exam}
\providecommand*{\examautorefname}{Example}

\theoremstyle{definition}
\newaliascnt{defi}{theo}
\newtheorem{defi}[defi]{Definition}
\aliascntresetthe{defi}
\providecommand*{\defiautorefname}{Definition}
\newaliascnt{nota}{theo}
\newtheorem{nota}[nota]{Notation}
\aliascntresetthe{nota}
\providecommand*{\notaautorefname}{Notation}

\addto\extrasenglish{%
%  \renewcommand{\sectionautorefname}{Section}
  \renewcommand{\subsectionautorefname}{\S}
  \renewcommand{\subsubsectionautorefname}{\S\S}%
}





%%% Famous group schemes
\DeclareMathOperator{\GL}{GL}
\DeclareMathOperator{\UU}{U}
\DeclareMathOperator{\SL}{SL}
\DeclareMathOperator{\SU}{SU}
\DeclareMathOperator{\Sp}{Sp}
\DeclareMathOperator{\SO}{SO}
\DeclareMathOperator{\matr}{M}


%% Common Operators

\DeclareMathOperator{\Gal}{\bf Gal}
\DeclareMathOperator{\Lie}{Lie}
\DeclareMathOperator{\Stab}{Stab}
\DeclareMathOperator{\irr}{Irr}
\DeclareMathOperator{\End}{End}
\DeclareMathOperator{\aut}{Aut}
\newcommand{\Span}{\operatorname{Span}}
\DeclareMathOperator{\spec}{Spec}

\renewcommand{\ker}{\mathrm{Ker}}
\newcommand{\coker}{\mathrm{Coker}}
\renewcommand{\hom}{\mathrm{Hom}}
\newcommand{\res}{\mathrm{Res}}
\newcommand{\ind}{\mathrm{Ind}}
\newcommand{\im}{\mathrm{Im}}
\newcommand{\Tr}{\mathrm{Tr}}
\newcommand{\rad}{\mathrm{Rad}}
\newcommand{\diag}{\mathrm{diag}}
\newcommand{\id}{\mathbf{1}}
\newcommand{\Ql}{\underline{\dbQ_\ell}}

\newcommand{\Ad}{\mathrm{Ad}}
\newcommand{\Cen}{\ensuremath{\mathrm{C}}}

\newcommand{\supp}{\mathrm{Supp}}
\newcommand{\cl}[1]{\mathrm{cl}\left(#1\right)}
%\newcommand{\cl}[1]{\overline{#1}}

\renewcommand{\L}{\mcal{L}}
%% Famous Fields, Rings, Sets etc.

\renewcommand{\O}{\varnothing}
\newcommand{\dbN}{\mathbb N}
\newcommand{\dbZ}{\mathbb Z}
\newcommand{\dbF}{\mathbb F}
\newcommand{\dbQ}{\mathbb Q}
\newcommand{\dbR}{\mathbb R}
\newcommand{\dbC}{\mathbb C}
\newcommand{\dbA}{\mathbb A}
\newcommand{\dbP}{\mathbb P}

\newcommand{\kk}{k}
\newcommand{\B}{\mcal{B}}

\newcommand{\frob}{\sigma}

%% Common Unary Functions
\newcommand{\gen}[1]{\langle{#1}\rangle}
\newcommand{\set}[1]{\left\{{#1}\right\}}
\newcommand{\norm}[1]{\left\|#1\right\|}
\newcommand{\abs}[1]{\left|#1\right|}
\newcommand{\inner}[1]{\left(#1\right)}


%% Shorter font named
\newcommand{\mcal}{\mathcal}
\newcommand{\mbf}{\mathbf}
\newcommand{\mfr}{\mathfrak}
\newcommand{\msf}{\mathsf}

\newcommand{\widebar}{\overline}
\renewcommand{\tilde}{\widetilde}

\newcommand{\loc}{\mathrm{loc}}

%% 

\newenvironment{sol}{\sc Solution. \rm}{\hfill \qedsymbol\bigskip}

\title{Generalized functions\\Tutorial notes}
\begin{document}\maketitle
\part*{Tutorial 4}\setcounter{section}{4}
\subsection{Fr\'chet spaces}
\begin{defi}
	\begin{enumerate}
		\item  Metrizable space;
		\item  $S_1$ or first countable, if every point has a countable basis;
		\item Fr\'echet =locally convex complete metrizable tvs.
	\end{enumerate}
\end{defi}
\begin{exam} $C^{\infty}(K)$ is Fr\`echet, but not normed (with the norm of uniform convergence of all derivatives). 
\end{exam}


\begin{exer}
	TFAE for a lctvs:
	\begin{enumerate}
		\item metrizability;\item  first countable; \item countable collection of seminorms
	\end{enumerate}
In particular: complete+either of the above gives Fr\`echet.
\end{exer}
%
%\begin{sol}
%	Metrizability implies first countability, since any point $x$ has $\set{B(x,\epsilon):\epsilon\in\dbQ_{>0}}$ as a neighborhood basis.
%	Assume $V$ is first countable ad let $\set{U_i}_{i=1}^\infty$ be a countable neighborhood base at $0$. Since $V$ is locally convex, we may replace each $U_i$ with $0\in U'_i\subseteq U_i$ open and convex, and, e.g. by taking $C_i=U_i\cap (-U_i)$, there exists a local basis $\set{C_i}_{i=1}^\infty$ of open, convex and balanced neighbourhoods of $0$. Put $\eta_i=N_{C_i}$. Note that $N_{C_i}(x)<1$ if and only if $x\in \alpha C$ for some $\alpha<1$, which, by convexity and balancedness, is equivalent to $x\in C$. In particular, we have that $C_i=\eta_{i}^{-1}(-\infty,1)$ is open in the topology induced from $\set{\eta_i}$. In particular, $\set{\eta_i}$ generate the same topology on $V$ as the $\set{C_i}$. 
%	
%	\begin{lem}
%		Let $\eta$ ve a seminorm on $V$. Define $d_\eta(x,y)=\frac{\eta(x-y)}{1+\eta(x-y)}$. Then $d_\eta$ is a translation invariant \emph{semi-metric}, i.e.\ it is non-negative, symmetric and satisfies the triangle inequality. Furthermore, $d_\eta(x,y)=0$ iff $\eta(x-y)=0$. 
%	\end{lem}
%	
%	\begin{proof}
%		Translation invariant, non-negative and symmetric are obvious. The triangle inequality follows from the convexity of the function $f(t)=\frac{t}{1+t}$ (i.e.\ $f(t+s)\le f(t)+f(s)$), and the triangle inequality for $\eta$; compute its second derivative and verify that its strictly negative for $t>0$. The last statement is obvious.
%	\end{proof}
%	
%	Finally, assume the topology on $V$ is generated by a family of seminorms $\set{\eta_i}_{i=1}^\infty$. Define $d(x,y)=\sum_{i=1}^n 2^{-i}d_{\eta_{i}}(x-y)$. One easily verifies that $d$ is a metric on $V$ (non-negativity, symmetry and the triangle inequality are obvious; for the positivity axiom note that given $x,y\in V$ we have that $d(x,y)=0$ iff $d_{\eta_i}(x,y)=0$, iff $\eta_i(x-y)=0$ for all $i$, iff $x-y$ is in the intersection of all open balls with respect to all seminorms on $V$, which can occur iff $x-y=0$). By definition, the metric $d$ is translation invariant, i.e.\ $d(x+z,y+z)=d(x,y)$ for all $x,y,z\in V$. In particular, to show that the two topologies coincide, it suffices to verify that admit mutually refining neighborhood bases at $0$; specifically- that any open ball with respect to $d$ is contained in an open set with respect to $\set{\eta_i}$, and vice versa.
%	
%	Note that $\eta_i(x)<\delta$ iff $d_{\eta_i}(x,0)<\frac{\delta}{1+\delta}$, so the $d_{\eta_i}$'s generate the same topology as the $\eta_i$'s. [This became boring, finish this.] 
%\end{sol}
\begin{exer}
	Show that $C^\infty(\mathbb {S}^1)\simeq \mathrm{SW}(\dbZ)$, where $\mathrm{SW}(\dbZ)$ denotes the space of rapidly decaying sequences $ \set{(x_n)_{n\in\dbZ}:\lim_{n\to\infty}\abs{x_n}n^\alpha<\infty\text{ for all }\alpha\in\dbN}$ indexed by $\dbZ$.
\end{exer}
\begin{sol}
	\newcommand{\F}{\mcal{F}}\newcommand{\dbS}{\mathbb S}
	Both spaces are Fre\`echet: $C^\infty)(\dbS^1)$ is endowed with the countable family of seminorms $\eta_j(f)=\sup_{x\in \dbS^1}\abs{f^{(j)}(x)}$ and $\mathrm{SW}(\dbZ)$ with the countable family $\nu_j((c_n)_n)=\sup_{n\in\dbZ}\abs{n^jc_n}$ (here $j\in\dbN$), and the Fourier transform map $\F:C^\infty(\dbS^1)\to \mathrm{SW}(\dbZ)$ defines a bijection of the two spaces. To prove $\mcal{F}$ is a homeomorphism, we need to prove that both $\F$ and $\F^{-1}$ are bounded with respect to these seminorms, i.e. for any $j$, there exists $M_1,M_2>0$ and $k_1(j),k_2(j)$ such that
	\[\nu_j(\F(f))\le M_1 \eta_{k_1(j)}(f)\quad\text{and}\quad\eta_j(\F^{-1}(c_n))\le M_2\eta_{k_2(j)}((c_n)).\]
	
	Recall that $\F$ is given by $\F(f)=\left(\frac{1}{2\pi}\int_{0}^{2\pi}f(x)e^{- n  xi}dx\right)_{n\in\dbZ}$, with inverse $\F^{-1}((c_n)_{n\in\dbZ})=\sum_{n=-\infty}^\infty c_ne^{inx}$. Let $\eta_j(f):=\sup_{x\in \dbS^1}\sup\abs{f^{(j)}(x)}$ and $\nu_j((c_n)_n)=\sup_{n\in\dbZ}\abs{n^{j}c_n}$.
	
	Given $f\in C^\infty(\dbS^1)$ with $\F(f)=(c_n)_{n\in\dbZ}$, and $k\in\dbZ$, we have that
	\begin{multline*}
	\abs{k^j c_k}=\abs{\gen{f,e^{-ikx}}}=\abs{\sum_{n\in\dbZ}\frac{1}{2\pi}\int_0^{2\pi}c_nn^j e^{i(n-k)x}dx}\\
	\le\frac{1}{2\pi}\int_0^{2\pi}\abs{\sum_{n\in\dbZ}c_nn^j e^{i(n-k) x}}dx=\frac{1}{2\pi}\int_0^{2\pi}\abs{f^{(j)}e^{-ikx}}dx\\
	\le \frac{1}{2\pi}\int_0^{2\pi}\sup_{x\in \dbS^1}\abs{f^{(j)}(x)}dx=\eta_j(f)
	\end{multline*}
	
	In particular, we have that $\nu_j(\mcal{F})\le \eta_j(f)$.
	On the other hand, we have that
	\begin{multline*}
	\eta_j(f)=\sup_{x\in\dbS^1}\abs{f^{(j)}(x)}=\sup_{x\in\dbS^1}\sum_{n\in\dbZ} c_n(ni)^j e^{inx}\le \\\sum_{n\in\dbZ\setminus{\set{0}}}\frac{1}{n^2}\abs{c_nn^{j+2}}\le\sup_{k\in\dbZ}\abs{k^{j+2}c_k}\sum_{n\in\dbZ\setminus{\set{0}}} \frac{1}{n^2}=\frac{2\pi^2}{6}\nu_{j+2}((c_n).
	\end{multline*}
%Let $B_\epsilon^{\eta_j}(0))=\set{f\in C^\infty(\dbS^1):\eta_j(f)<1}$. Then 
%	\[\F(B_\epsilon^{\eta_j}(0))=\set{(c_n)_{n\in\dbZ}:\abs{\F^{-1}((c_n n^{-j})_n)(x)}=\abs{\sum_n c_nn^{-j}e^{inx}}<\infty\text{ for all x}}.\]
%	Indeed, the $n$-th Fourier coefficient of $f^(j)$ is $c_n n{^-j}$ (follows from change of variable).
	\end{sol}

\begin{theo}[Banach-Steinhaus for Fr\`echet spaces] Let $X$ be a Fr\`echet space and $Y$ a normed space, and $H$ a family of continuous linear maps form $X$ to $Y$. Assume $\sup_{f\in H}\norm{f(x)}<\infty$ for any $x\in X$. Then the family $H$ is equicontinuous, i.e. for any $\epsilon>0$ there exists $\delta>0$ such that $\norm{f(x_1)-f(x_2)}<\epsilon$ for any $f\in F$ and $x_1,x_2\in X$ with $d(x_1,x_2)<\delta$. 
\end{theo}

\begin{exer} Show that $C^{-\infty}(\dbR)$ is weakly sequentially complete.
\end{exer}
\begin{proof}
	Let $(\xi_n)$ be a weakly Cauchy sequence in $(C_c^{\infty}(\dbR))^*$. We have a natural candidate for the limit of $(\xi_n)$, which is defined by
	\[\gen{\xi,f}:=\lim_{n\to\infty}\gen{\xi_n,f},\]
	which exists by definition of weakly Cauchy. $\xi$ is clearly linear, but not obviously continuous.
	
	Let $K\subseteq\dbR$ be a compact set. We consider the restrictions of the $\xi_n$'s and $\xi$ to $C^\infty(K)$. More specifically, we consider $F=\set{\xi_n\mid_{C^\infty(K)},\xi\mid_{C^\infty(K)}}\subseteq\hom(C^\infty(K),\dbR)$. By weak convergence, $F$ is pointwise bounded, and hence, by Banach-Steinhaus, it is equicontinuous. Let $f_n$ be a sequence in $C^\infty(K)$, converging to $f$ (note that $C^\infty(K)$ is sequentially complete with respect to the topology of uniform convergence of all derivatives, so $f\in C^\infty(K)$). Then $d(f_n,f)\xrightarrow{n\to\infty}0$, and hence given $\epsilon>0$, for any $n\gg 0$ we have that $d(f_n,f)<\delta$ and hence \[\abs{\gen{\xi,f-f_n}}\le \sup_{\nu\in F}\abs{\gen{\nu,f-f_n}}<\epsilon\]
 	for all $n\gg 0$.
 \end{proof}

As seen in the lecture, we have a description of $C_c^\infty(\dbR)$ as a direct limit of Fr\`echet spaces: 
\[C_c^\infty(\dbR)=\varinjlim_{K\subseteq \dbR\text{ compact}}C_K^\infty(\dbR).\]
The following exercise explicitly describes the topology of this space.
\begin{exer}
	Given $n\in\dbN$, $k_n\in\dbZ_{\ge 0}$ and $\epsilon_n>0$, put 
	\[A_{k_n,\epsilon_n}=\set{f\in C^\infty(\dbR):\supp(f)\subseteq[-n,n]\text{ and }\sup_{x\in\dbR}\abs{f^{(k_n)}(x)}<\epsilon_n},\]
	the open ball of radius $\epsilon_n$ with respect to $\nu_{k_n}(f)=\sup\abs{f^{(k_n)}}$ on the space $C_{[-n,n]}^\infty(\dbR)$.
	Define
		\[U_{(k_n,\epsilon_n)_n}:=\sum_{n\in\dbN} A_{k_n,\epsilon_n},\]
	where $(k_n,\epsilon_n)$ is a sequence of pairs in $\dbZ_{\ge 0}\times \dbR_{>0}$.
	 
	Show that a sequence $(f_n)$ in $C_c^\infty(\dbR)$ converges to $f$ with respect to the topology generated by the $U_{(k_n,\epsilon_n)}$'s if and only if it converges in the sense defined in the first lecture.
\end{exer}
\begin{sol}
	Let $(f_n)_n$ be a sequence in $C_c^\infty(\dbR)$. Assume $f_n\to f$ in the sense defined previously, so that there exists $K\subseteq \dbR$ such that $\supp(f_n)\cup\supp(f)\subseteq K$ and $f_n^{(k)}\to f^{(k)}$ uniformly on $K$ for all $k\in\dbN$. In particular, given $k\in\dbN$ and $\epsilon>0$, it holds that $\sup_{x\in K}\abs{f^{(k)}_n(x)-f^{k}(x)}<\epsilon$ for all but finitely many $n$'s. 
	
	Let $U=U_{(k_n,\epsilon_n)}$ be an open set as above, and let $n_0$ be large enough so that $K\subseteq [-n_0,n_0]$. Then, by the previous paragraph, we have that $\norm{f^{(k_{n_0})}_m-f^{(k_{n_0})}}_\infty<\epsilon_{n_0}$, and thus $f_m-f\in A_{k_{n_0},\epsilon_{n_0}}\subseteq U$ for all but finitely many $m$'s. Therefore, we have that $(f_n)$ converges to $f$ in the given topology.
	
	Conversely, assume we have convegence in topology of $f_n$ to $f$, and let $k\in\dbN$ and $\epsilon>0$ be arbitrary. We want to show $\norm{f^{(k)}_n-f^{(k)}}_{\infty}<\epsilon$ for all but finitely many $n$'s. To show this, it is (more than) enough to consider the set $U_{(k_n,\epsilon_n)}$, with $k_n=k$ and $\epsilon_n=\epsilon$ for all $n$, and use the convergence in topology to obtain that all but finitely many elements of the sequence $(f_n-f)$ are in this set and therefore, in particular, satisfy the desired inequality.
\end{sol}
\begin{rem}
	One can also show that the topology on $C_c^\infty(\dbR)$ is generated by the convex hulls of the $U_{(k_n,\epsilon_n)}$, thereby giving a direct proof of local convexity.
\end{rem}
	
	\begin{exer}
		Let $S\in C_c^\infty(\dbR)$ be bounded. Then there exists $K$ compact such that $S\subseteq C^\infty_K(\dbR)$. 
	\end{exer}

\begin{sol}
	We need the following lemma:
	\begin{lem}
		Let $V$ be a locally convex topological vector space. A set $S\subseteq V$ is bounded if and only if $\eta(S)<\infty$ for any continuous seminorm $\eta$ on $V$.
	\end{lem}
	\begin{proof}
		If $S$ is bounded and $\eta$ is a countinuous seminorm with $B=\eta^{-1}(-\infty,1)$, the open ball around $0$ of radius $1$, then there exists $\lambda>0$ such that $S\subseteq \lambda B$ and hence $\eta(S)\subseteq \eta(\lambda B)\subseteq [0,\lambda]$. 
		
		Conveserly, if $S$ is bounded with respect to all continuous seminorms, let $0\in U$ be open and let $0\in C\subseteq U$ be a OCB set. Put $\lambda=\sup N_C(S)<\infty$, then $S\subseteq \lambda C\subseteq \lambda U$. 
	\end{proof}
	Assume $S$ is not included in $C_K^\infty(\dbR)$ for any compact $K$. In particular, this means that there exists a sequence $(f_n)$ of elements of $S$ and a sequence $(x_n)$ in $\dbR$ without accumulation points such that $f_n(x_n)\ne 0$ for all $n\in\dbN$. Wlog, we may assume $x_n\in [-n,n]$ for all $n\in\dbN$. Define
	\[\eta(f)=\sup_{n\in\dbN}\frac{n\abs{f(x_n)}}{\abs{f_n(x_n)}}.\]
	Then $\eta$ is continuous (home exercise), because $\eta(f)\le \alpha_n \norm{f}_\infty$ for any $f\in C_c^\infty(\dbR)$ for suitable $\alpha_n$ dependent on $\supp{f}$, and $\eta(S)$ is unbounded, because $\eta(f_n)=n$ for any $n\in\dbN$.
\end{sol}

\subsection{Topologies on $C^{-\infty}(\dbR)$}
\begin{defi}
	Let $V$ be a topologival vector space, and $V^*$ its continuous dual. Given $S\subseteq V$ and $\epsilon>0$, define $U_{S,\epsilon}:=\set{\xi\in V^*: \forall f\in S,\:\abs{\gen{\xi,f}}<\epsilon}$.
	\begin{enumerate}
		\item A set $B\subseteq V$ is bounded if for any open set $0\in U\subseteq V$, there exists $\lambda\in\dbR$ such that $B\subseteq\lambda U$.
		\item The \emph{weak}(-$*$) topology on $V^*$ has as a neighborhood basis at $0$ the collection $\mcal{B}_w:=\set{U_{\epsilon,S}: \epsilon>0\text{ and }S\text{ finite}}$.
		\item  The \emph{strong} topolgy has as a neighborhood basis at $0$ the collection $\mcal{B}_s:=\set{U_{\epsilon,S}:\epsilon>0\text{ and } S\text{ bounded}}$.
	\end{enumerate}
\end{defi}

\begin{exer}Prove the following variant of the Banach-Steinhaus Theorem.
	
	Let $X$ be a Fr\`echet space and $Y$ a normed space, and let $H$ be a family of bounded linear operators. Let $\set{\eta_i}_{i=1}^\infty$ be a family of seminorms generating the topology on $X$.  Assume $\sup_{f\in H}\norm{f(x)}<\infty$ for all $x\in X$. Then $\sup_{f\in H,\: \eta_i(x)=1}\norm{f(x)}<\infty$ for all but finitely many $i$'s.
\end{exer}

\begin{sol}
	
\end{sol}


\begin{exer}
	Prove that the inclusion $f\mapsto\xi_f:C_c^\infty(\dbR)\to C^{-\infty}(\dbR)$ is dense with respect to the \emph{strong} topology.
\end{exer}
\begin{rem}Recall that we considered the weak topology in the first tutorial. The following will give an alternative proof of this fact.
\end{rem}
\begin{sol}
	Let us first show that this inclusion is dense in $C_c^{-\infty}(\dbR)$, the space of compactly supported distributions. Following this we will show that $C_c^{-\infty}(\dbR)$ is dense in $C^{-\infty}(\dbR)$ in both topologies.
	
	Recall that, given $\xi\in C^{-\infty}$ and $\psi\in C_c^\infty(\dbR)$, we defined $\xi\ast \psi(x)=\gen{\xi,L_x\bar{\psi}}$. Moreover, assuming $\supp(\xi)$ is compact, we have that $\xi\ast\psi\in C_c^\infty(\dbR)$. Essentially, by design and continuity of $\xi$, we have that $\xi_{\xi\ast \psi}=\xi\ast\xi_\psi$ (here the RHS is the distribution of two compactly supported distributions), i.e.
	\[\gen{\xi\ast\xi_\psi,f}=\gen{\xi,\widebar{\xi_\psi\ast \widebar{f}}}=\int_{\dbR}\xi\ast\psi(x)f(x)dx.\]
	[FIND REFERENCE.]

	Let $\xi\in C_c^{-\infty}(\dbR)$, and let $\psi_n$ be an approximation of unity with $\supp\psi_n\subseteq [-1/n,1/n]$. We will show that $\xi\ast\xi_{\psi_n}\to \xi$ both in the weak and the strong topology.
	
	\begin{itemize}
		\item \underline{Weakly}: We need only to show that $\gen{\xi\ast\xi_{\psi_n},f}\to\gen{\xi,f}$ as $n\to \infty$. This holds since $\gen{\xi\ast\xi_{\psi_n},f}=\gen{\xi,\widebar{\psi_n\ast\widebar{f}}}$ and $\psi_n\ast f$ converges to $f$ in $C_c^\infty(\dbR)$ (proof uses that $\psi_n\ast f$ is supported on $[1,1]+\supp (f)$ and $(\psi_n\ast f)^{(k)}=\psi_n\ast f^{(k)}$ converges to $f^{(k)}$ uniformly). 
		
		\item \underline{Strongly}: This is a bit harder; it essentially uses the fact (Lagrange MVT) that the $k+1$-th derivative gives bounds on the value of the $k$-derivative. Fix a set $U_{S,\epsilon}=\set{\xi:\forall f\in S, \abs{\gen{\xi,f}}<\epsilon}$ as above with $S\subseteq C_c^{\infty}(\dbR)$ bounded. We need to prove $()\xi\ast\psi_n-\xi)\in U_{S,\epsilon}$ for all but finitely many $n$'s. 
		
		Let's set up some notation: $\xi_n=\xi\ast\psi_n$ and $\norm{f}_k=\sup_{x\in\dbR}\abs{f^{(k)}}$. By continuity of $\xi$, we have that
		\[\abs{\gen{\xi-\xi_n,f}}=\abs{\xi,f-\widebar{\psi_n\ast\widebar{f}}}\le C\norm{f-\widebar{\psi_n\ast\widebar{f}}}_k\]
		for some $k$ and $C>0$. 
		
		Using Lagrange's MVT, we have that
		\begin{multline*}
			\abs{f^{(k)}(x)-\widebar{\psi_n\ast \widebar{f^{(k)}}}(x)}=\int_{-1/n}^{1/n}(f^{(k)}(x)-f^{(k)}(x+t))\psi_n(t)dt\\\le \sup_{\abs{t}\le 1/n}\abs{f^{(k)}(x)-f^{(k)}(x+t)}
			\le \sup_{\abs{t}\le 1/n}\abs{f^{(k+1)}(c)t}\le\frac{\norm{f}_{k+1}}{n}
		\end{multline*}
		for some $c\in[x,x+t]$. By boundedness of $S$, there exists $\lambda>0$ such that $S\subseteq \lambda B_{\norm{\cdot}_{k+1}}(0,1)$. In particular, if $f\in S$ we have that $\norm{f}_{k+1}\le \lambda$, so that
		\[\abs{f^{(k)}(x)-\widebar{\psi_n\ast\widebar{f^{(k)}}}(x)}\le \frac{\lambda}{n}.\]
		In particular, taking $n\gg 0$, we have that the RHS is smaller than epsilon for all $f\in S$, and hence $\xi_n\in U_{S,\epsilon}$. 
		
		Finally, since the weak topology is coarser than the strong topology, it suffices to show that $C_c^{-\infty}(\dbR)$ is strongly dense in $C^{-\infty}(\dbR)$. This follows from a previous exercise, namely, given $S\subseteq C_c^\infty(\dbR)$ bounded, there exists $K\subseteq\dbR$ compact such that $S\subseteq C_K^\infty(\dbR)$. Then, given $\xi\in C^{-\infty}(\dbR)$, we have that $\xi\mid_S\equiv (\xi\cdot I_K)_S$, where $I_K$ is the indicator of $K$, so that $\xi\cdot I_K\in C^{-\infty}_c(\dbR)$ is an element of $\xi+U_{S,\epsilon}$ for all $\epsilon>0$. 
	\end{itemize}
\end{sol}
\begin{exer}
\begin{enumerate}
	\item  Show that $C^{-\infty}(\dbR)$ is not complete in the weak topology.
	\item Show that the weak completion of $C^{-\infty}(\dbR)$ is $(C_c(\dbR))^\sharp$, the abstract dual.
	\item  Show that $C^{-\infty}(\dbR)$ is \emph{strongly} complete. 
\end{enumerate}	
\end{exer}

\begin{sol}
	For (1), define the waek topology on $(C_c^\infty(\dbR))^\sharp$ (makes sense), and show that the embedding $C^{-\infty}(\dbR)\to(C_c^\infty(\dbR))^\sharp$ is strict with dense image.
	
	Try to prove the universal property for (2). 
	
	(3)?
\end{sol}
\begin{defi}
	Given a closed subspace $W\subseteq \dbR^n$  and $m\in\dbN$ define
	\[V_m(C_c^\infty(\dbR^n),W):=\set{f\in C_c^\infty(\dbR^n):\frac{\partial^\alpha}{(\partial x)^\alpha}f\mid_W\equiv0,\abs{\alpha}\le m}.\]
	Defined similarly for $W$ a subset.
\end{defi}
\begin{exer}
	Let $W$ be a $k$-dimensional subspace of $\dbR^n$ and $U=\dbR^n\setminus W$. Show that 
	\[\widebar{C_c^\infty(U)}=\bigcap_{m=0}^\infty V_m(C_c^\infty(\dbR^n),W).\]
\end{exer}
\begin{rem}This exercise is true for a general open $U$.
\end{rem}
\begin{sol}
	\begin{itemize}
		\item[$\subseteq$] Enough to show $C_c^\infty(U)\subseteq \bigcap V_m$, since the RHS is clearly closed. Let $f\in C_c^\infty(U)$. Then $f$ vanishes on $W$, and in particular so do all partial derivarives of $f$ of any order, at any point $x\in W$. 
		\item[$\supseteq$] Take $f\in \bigcap V_m$, we want to show that $f$ is the limit of a sequence of functions with support on $U$. We can use cutoff functions, which are identically zero in a small neighbourhood of $W$, and put $f_n=f\cdot I_n$. 
	\end{itemize}
\end{sol}

\begin{defi}
	Define $F_m((C_c(\dbR^n)^*,W):=\set{\xi\in C_c^\infty(\dbR^n)^*:\xi\mid_{V_m}\equiv 0}$, where $V_m$ is as above.
\end{defi}
\begin{exer}
	Show that $C_W^{\infty}(\dbR^n)\ne \cup_m F_m$. 
\end{exer}
\begin{sol}Take a comb of derivatives of $\delta$ at a discrete set of points. 
\end{sol}

\begin{exer}
	Prove that for any $U\subseteq\dbR^n$ with compact closure and any $\xi\in C^{-\infty}_W(\dbR^n)$, there exists $\xi'\in F_m$ such that $\xi\mid_U\equiv\xi'\mid_U$. 
\end{exer}

\begin{exer}
	Compute $\widebar{C_c^\infty(\dbR^n\setminus\set{0})}$.
\end{exer}
\end{document}