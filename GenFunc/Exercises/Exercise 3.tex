\documentclass[11pt, reqno,a4paper, twoside]{amsproc}
\usepackage{ifpdf}
\usepackage[english]{babel}
\usepackage{amsmath}
\usepackage{amsthm, amssymb,bm}
\usepackage[margin=1.4 cm, top=0cm]{geometry}
%usepackage{fullpage}
\usepackage{ucs}
\usepackage{tikz}
\usepackage{hyperref}
\usetikzlibrary{matrix}
\usepackage{mathrsfs}
\usepackage{eucal}
\hypersetup{
  colorlinks=true,
  citecolor=black,
  linkcolor=black,
  urlcolor=black,
  filecolor=red}
  
  
	
\usepackage{aliascnt}
\numberwithin{equation}{section}


\newtheorem{mainthm}{Theorem}\renewcommand{\themainthm}{\Roman{mainthm}}

\newtheorem{theo}{Theorem}[section]	
\newtheorem*{theo*}{Theorem}

\newaliascnt{lem}{theo}
\newtheorem{lem}[lem]{Lemma}
\aliascntresetthe{lem}

\newaliascnt{propo}{theo}
\newtheorem{propo}[propo]{Proposition}
\aliascntresetthe{propo}

\newaliascnt{corol}{theo}
\newtheorem{corol}[corol]{Corollary}
\aliascntresetthe{corol}

\newaliascnt{ques}{theo}
\newtheorem{ques}[ques]{Question}
\aliascntresetthe{ques}

\newaliascnt{conj}{theo}
\newtheorem{conj}[conj]{Conjecture}
\aliascntresetthe{conj}


\newaliascnt{exer}{theo}
\newtheorem{exer}[exer]{Exercise}
\aliascntresetthe{exer}

\newaliascnt{assumption}{theo}
\newtheorem{assumption}[assumption]{Assumption}
\aliascntresetthe{assumption}

\providecommand*{\mainthmautorefname}{Theorem}
\providecommand*{\theoautorefname}{Theorem}
\providecommand*{\propoautorefname}{Proposition}
\providecommand*{\lemautorefname}{Lemma}
\providecommand*{\corolautorefname}{Corollary}
\providecommand*{\quesautorefname}{Question}
\providecommand*{\assumptionautorefname}{Assumption}
\providecommand*{\conjautorefname}{Conjecture}
\providecommand*{\exerautorefname}{Exercise}


\theoremstyle{remark}

\newaliascnt{rem}{theo}
\newtheorem{rem}[rem]{Remark}
\aliascntresetthe{rem}
\providecommand*{\remautorefname}{Remark}
\newtheorem*{claim}{Claim}
\newaliascnt{exam}{theo}
\newtheorem{exam}[exam]{Example}
\aliascntresetthe{exam}
\providecommand*{\examautorefname}{Example}

\theoremstyle{definition}
\newaliascnt{defi}{theo}
\newtheorem{defi}[defi]{Definition}
\aliascntresetthe{defi}
\providecommand*{\defiautorefname}{Definition}
\newaliascnt{nota}{theo}
\newtheorem{nota}[nota]{Notation}
\aliascntresetthe{nota}
\providecommand*{\notaautorefname}{Notation}

\addto\extrasenglish{%
%  \renewcommand{\sectionautorefname}{Section}
  \renewcommand{\subsectionautorefname}{\S}
  \renewcommand{\subsubsectionautorefname}{\S\S}%
}





%%% Famous group schemes
\DeclareMathOperator{\GL}{GL}
\DeclareMathOperator{\UU}{U}
\DeclareMathOperator{\SL}{SL}
\DeclareMathOperator{\SU}{SU}
\DeclareMathOperator{\Sp}{Sp}
\DeclareMathOperator{\SO}{SO}
\DeclareMathOperator{\matr}{M}


%% Common Operators

\DeclareMathOperator{\Gal}{\bf Gal}
\DeclareMathOperator{\Lie}{Lie}
\DeclareMathOperator{\Stab}{Stab}
\DeclareMathOperator{\irr}{Irr}
\DeclareMathOperator{\End}{End}
\DeclareMathOperator{\aut}{Aut}
\newcommand{\Span}{\operatorname{Span}}
\DeclareMathOperator{\spec}{Spec}

\renewcommand{\ker}{\mathrm{Ker}}
\newcommand{\coker}{\mathrm{Coker}}
\renewcommand{\hom}{\mathrm{Hom}}
\newcommand{\res}{\mathrm{Res}}
\newcommand{\ind}{\mathrm{Ind}}
\newcommand{\im}{\mathrm{Im}}
\newcommand{\Tr}{\mathrm{Tr}}
\newcommand{\rad}{\mathrm{Rad}}
\newcommand{\diag}{\mathrm{diag}}
\newcommand{\id}{\mathbf{1}}
\newcommand{\Ql}{\underline{\dbQ_\ell}}

\newcommand{\Ad}{\mathrm{Ad}}
\newcommand{\Cen}{\ensuremath{\mathrm{C}}}

\newcommand{\supp}{\mathrm{Supp}}
\newcommand{\cl}[1]{\mathrm{cl}\left(#1\right)}
%\newcommand{\cl}[1]{\overline{#1}}

\renewcommand{\L}{\mcal{L}}
%% Famous Fields, Rings, Sets etc.

\renewcommand{\O}{\varnothing}
\newcommand{\dbN}{\mathbb N}
\newcommand{\dbZ}{\mathbb Z}
\newcommand{\dbF}{\mathbb F}
\newcommand{\dbQ}{\mathbb Q}
\newcommand{\dbR}{\mathbb R}
\newcommand{\dbC}{\mathbb C}
\newcommand{\dbA}{\mathbb A}
\newcommand{\dbP}{\mathbb P}

\newcommand{\kk}{k}
\newcommand{\B}{\mcal{B}}

\newcommand{\frob}{\sigma}

%% Common Unary Functions
\newcommand{\gen}[1]{\langle{#1}\rangle}
\newcommand{\set}[1]{\left\{{#1}\right\}}
\newcommand{\norm}[1]{\left\|#1\right\|}
\newcommand{\abs}[1]{\left|#1\right|}
\newcommand{\inner}[1]{\left(#1\right)}


%% Shorter font named
\newcommand{\mcal}{\mathcal}
\newcommand{\mbf}{\mathbf}
\newcommand{\mfr}{\mathfrak}
\newcommand{\msf}{\mathsf}

\newcommand{\widebar}{\overline}
\renewcommand{\tilde}{\widetilde}

%% Acronyms

\newcommand{\lctvs}{locally convex topological vector space}
\newcommand{\ocb}{open, balanced, and convex}
%% 
\renewcommand{\thesubsection}{Exercise~\arabic{subsection}}
\title{Generalized functions\\Exercise sheet 3}
\begin{document}\maketitle

\subsection{}
Given a compact subset $K\subseteq\dbR$, $k\in\dbZ_{\ge 0}$ and $\epsilon>0$, put \[B^K_{\epsilon,k}:=\set{f\in C^\infty_c(\dbR):\supp(f)\subseteq K,\: \sup_{x\in\dbR}\abs{f^{(k)}(x)}<\epsilon}.\]

\begin{enumerate}
	\item Let $I$ denote the set of sequences $(\epsilon_n,k_n)_{n=1}^\infty$, with $\epsilon_n>0$ and $k_n\in\dbZ_{\ge 0}$. Show that the following collections generate the same topology on $C_c^\infty(\dbR)$:
	\begin{itemize}
		\item $\mfr{T}_1=\set{U_{(\epsilon_n,k_n)}}_{(\epsilon_n,k_n)\in I}$ where $U_{(\epsilon_n,k_n)}:=\sum_{n\in\dbN} B^{[-n,n]}_{\epsilon_n,k_n}$; and
		\item $\mfr{T}_2:=\set{V_{(\epsilon_n,k_n)}}_{(\epsilon_n,k_n)\in I}$ where $V_{(\epsilon_n,k_n)}:=\mathrm{conv}\left(\bigcup_{n\in\dbN}B_{\epsilon_n,k_n}^{[-n,n]}\right)$.
	\end{itemize}
	Recall that $\sum_{n\in\dbN}X_n:=\set{\sum_{n\in\dbN}x_n:x_n\in X_n\text{ and $x_n=0$ for }n\gg 0}$, for $X_1,X_2,\ldots$ subsets of a vector space.
	\item Show that a sequence $(f_n)_{n=1}^\infty$ in $C_c^\infty(\dbR)$ converges to $f$ in the sense defined in the first lecture if and only if it converges to $f$ with respect to the topology generated by the collections defined in the previous exercise.
\end{enumerate} 

\subsection{}
	Recall that, given a topological vector space $V$, we write $V^*$ and $V^\sharp$ to denote the continuous and full dual spaces of $V$, respectively. 
	\begin{enumerate}
		\item Let $V$ be a Fr\`echet space. Show that given a finite linearly independent set $\set{v_1,\ldots,v_n}$ and values $\lambda_1,\ldots,\lambda_n\in\dbR$, there exists $\xi\in V^*$ such that $\gen{\xi,v_i}=\lambda_i$ for all $i=1,\ldots,n$. 
		\item Show that $(C_c^\infty(\dbR))^*$ is dense in $(C_c^\infty(\dbR))^\sharp$ with respect to the weak topology.
		
		\textit{Remark}. The weak topology on $V^\sharp$ is defined similarly to $V^*$; describe a generating set for this topology as part of the exercise.
		
		\item  Conclude that $(C_c^\infty(\dbR))^*$ is not complete with respect to weak topology.
		
		\item  * Show that $(C_c^\infty(\dbR))^\sharp$ is complete with respect to weak topology. Conclude that it is the weak completion of $(C_c^\infty(\dbR))^*$
	\end{enumerate}

\subsection{} Recall that given a subspace $W$ of $\dbR^n$ and $m\in\dbZ_{\ge 0}$ we defined
\[V_m(C_c^\infty(\dbR^n),W)=\set{f\in C_c^\infty(\dbR^n):\frac{\partial^\alpha}{(\partial x)^\alpha}f\mid_W\equiv 0\text{ for any $\alpha$ with }\abs{\alpha}\le m},\]
and 
\[F_m((C_c^\infty(\dbR^n))^*,W)=\set{\xi\in (C_c^\infty(\dbR^n))^*:\xi\mid_{V_m}\equiv 0}.\]
\begin{enumerate}
	\item Show that, $\widebar{C_c^\infty(\dbR^n\setminus W)}=\bigcap_{m=0}^\infty V_m$. 
	\item Compute $\widebar{C_c^\infty(\dbR^n\setminus\set{0}))}$. 
	\item Show that $\bigcap_{m=0}^\infty F_m\ne (C_{W}^\infty(\dbR^n))^*$.
	\item Let $\varphi:\dbR^n\to\dbR^n$ be a diffeomorphism such that $\varphi(W)\subseteq W$. Prove that $F_m$ is invariant under the map $\xi\mapsto\varphi^*(\xi)$, where $\gen{\varphi^*(\xi),f}=\gen{\xi,f\circ\varphi}$. 
\end{enumerate}

\subsection{}Tensor products?
\end{document}
 